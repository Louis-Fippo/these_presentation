\def\Sceq#1#2{\mathbf{Sce}(#1 \reach #2)}
\def\live#1{\rho (#1)}
\def\pacc{\mathcal{R}}
\def\mypropp{\mathcal{P}}
\def\mypropq{\mathcal{Q}}
\def\myqprop#1{\mathcal{Q} (#1)}
\def\condnec{\mathcal{CN}}
\def\condsuf{\mathcal{CS}}
\def\liminf#1{\mathsf{linf}_{#1}}
\def\limsup#1{\mathsf{lsup}_{#1}}
\def\binf#1{\mathsf{binf}_{#1}}
\def\bsup#1{\mathsf{bsup}_{#1}}
\def\mystruct{\mathcal{G}}
\def\aG{\mathcal{G}}
\def\myG{\aG^\w_\ctx}
\def\cwG{\sat{\aG_\ctx^\w}}
\def\mycwG#1#2{\sat{\aG_{#1}^{#2}}}

\def\san{\mathcal{SAN}}
\def\an{\mathcal{N}}
\def\moy{moy}
\def\Cyc{\mathbf{Cyc}}
\def\C{\mathcal{C}}
\def\n{\eta}

%fonction indicatrice
\def\indica#1{\mathds{1}_{#1}}



%
\newcommand{\ANhitter}{\mathsf{frappeur}}
\newcommand{\ANtarget}{\mathsf{cible}}
\newcommand{\ANbounce}{\mathsf{bond}}
\newcommand{\ANinvariant}{\mathsf{invariant}}
\newcommand{\ANsort}{\mathsf{automate}}


\newcommand{\ansort}[1]{\ANsort(#1)}
\newcommand{\ansorte}{\sort}
\newcommand{\ANsorts}{\mathsf{automates}}
\newcommand{\ansortes}[1]{\PHsorts(#1)}
\newcommand{\ansorts}{\sortes}

\newcommand{\anhitter}[1]{\ANhitter(#1)}
\newcommand{\antarget}[1]{\ANtarget(#1)}
\newcommand{\anbounce}[1]{\ANbounce(#1)}
\newcommand{\anfrappeur}[1]{\ANhitter(#1)}
\newcommand{\ancible}[1]{\ANtarget(#1)}
\newcommand{\anbond}[1]{\ANbounce(#1)}
\newcommand{\aninvariant}[1]{\ANinvariant(#1)}


%macros pour les probas dans un glc
%pour les probas des différents types de noeuds dans le lcg
\def\GProbObj#1#2#3#4{\Prob{#1_#4,#2,#3} \DEF \Prob{obj_#4} [ \frac{\sum_{#1_{#4-1,i} \in Sol_#4} \Prob{#1_{#4-1,i},#2,#3}}{\card{Sol_#4}} ]}
\def\GProbSol#1#2#3#4{\Prob{#1_#4,#2,#3} \DEF  \prod_{#1_{#4-1,i} \in LS_#4} \Prob{#1_{#4-1,i},#2,#3} }
\def\GProbSyn#1#2#3#4{\Prob{#1_#4,#2,#3} \DEF  \prod_{#1_{#4-1,i} \in LS_#4} \Prob{#1_{#4-1,i},#2,#3} }
\def\GProbLS#1#2#3#4{\Prob{#1_#4,#2,#3} \DEF  \sum_{#1_{#4-1,i} \in Obj_#4} \indica{\ctx} (\antarget{#1_{#4-1}}) \cdot \Prob{#1_{#4-1,i},#2,#3} }
%\def\PHobjp#1#2#3{\PHobj{{#1}_{#2}}{{#1}_{#3}}}
\def\tranp#1#2#3{\pi^{#1} (#2,#3)}
%nouvelle macros pour les probas des objectifs
\def\GProbObjn#1#2#3#4{\Prob{#1_#4,#2,#3} \DEF \frac{\sum_{#1_{#4-1,i} \in Sol_#4} r_{#4-1} \cdot \Prob{#1_{#4-1,i},#2,#3}}{\sum_{#1_{#4-1,i} \in Sol_#4} r_{#1_{#4-1}} \cdot \Prob{#1_{#4-1,i},#2,#3}+ {\color{red}\sum_{j \in \bar{Sol_#4}} r_j} } \cdot \ppexit{A_{\antarget{#1_#4}}}{*}{t} }  


%macros pour les délais dans un glc
\def\GTimeObj#1#2#3#4{\Time{#1_#4,#2,#3} \DEF \Time{obj_#4} + [ \moy_{#1_{#4-1,i} \in Sol_#4} \Time{#1_{#4-1,i},#2,#3} ]}
\def\GTimeSol#1#2#3#4{\Time{#1_#4,#2,#3} \DEF  \moy_{#1_{#4-1,i} \in LS_#4} \Time{#1_{#4-1,i},#2,#3} }
\def\GTimeSyn#1#2#3#4{\Time{#1_#4,#2,#3} \DEF  \moy_{#1_{#4-1,i} \in LS_#4} \Time{#1_{#4-1,i},#2,#3} }
\def\GTimeLS#1#2#3#4{\Time{#1_#4,#2,#3} \DEF  \sum_{#1_{#4-1,i} \in Obj_#4} \indica{\ctx} (\antarget{#1_{#4-1}}) \cdot \Time{#1_{#4-1,i},#2,#3} }


%macros pour les traces
\def\tr#1{\mathsf{tr} (#1)}
\def\Automatet{\mathcal{A}}

%macros pour les motifs valides
\def\validMotif{\mathsf{validMotif}}
\def\Nsol{\textbf{Nsol}}
\def\transup#1{\xrightarrow{#1}}

