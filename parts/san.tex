% Définition du Process Hitting + sortes coopératives

%san presentation
\begin{comment}
\begin{frame}
\frametitle{Biological networks modelling}
\begin{tikzpicture}
\node (brn) at (0,5) {\begin{tikzpicture}[auto,scale=0.9] 
                                                 \path[use as bounding box] (-0.7,-0.3) rectangle (2.5,2);

                                                              \node[qgre,scale=0.8] (a) at (0,2) {a};
                                                              \node[mod,scale=0.8] (i) at (1,1) {i};
                                                              \node[qgre,scale=0.8] (b) at (0,0) {b};
                                                              \node[qgre,scale=0.8] (c) at (2,1) {c};
                                                              \path
                                                                (a) edge[inh,scale=0.1] (i)
                                                                (b) edge[act,scale=0.1] (i)
                                                                (i) edge[st,scale=0.1]  (c);
                                                              
                                                 \end{tikzpicture}};
\node[scale=0.6] (an) at (7,5) {\begin{tikzpicture}
\exandef
\end{tikzpicture}};
\node[scale=0.6] (sg) at (2,1) {
\begin{tikzpicture}[line join=bevel,font=\LARGE]
%%
  \node (6) at (405.0bp,18.0bp) [reach,nd4] {$\langle a_2,b_1,c_0\rangle$};
  \node (2) at (405.0bp,90.0bp) [reach,nd3] {$\langle a_2,b_0,c_0\rangle$};
  \node (1) at (570.0bp,90.0bp) [reach,nd1] {$\langle a_1,b_0,c_0\rangle$};
  \node (0) at (469.0bp,162.0bp) [reach,nd0] {$s_0 = \langle a_0,b_0,c_0\rangle$};
  \node (4) at (469.0bp,234.0bp) [reach,nd5] {$\langle a_0,b_1,c_0\rangle$};
 
 
  \draw [->,arc3] (0) ..controls (445.65bp,135.46bp) and (435.95bp,124.86bp)  .. (2);
  \draw [->,arc5] (0) ..controls (475.71bp,188.03bp) and (475.94bp,197.36bp)  .. (4);
  \draw [->,arc4] (2) ..controls (405.0bp,63.983bp) and (405.0bp,54.712bp)  .. (6);
  \draw [->,arc1] (0) ..controls (499.93bp,134.79bp) and (517.41bp,122.58bp)  .. (1);
  \draw [->,arc2] (1) ..controls (539.0bp,117.26bp) and (521.52bp,129.47bp)  .. (0);
  \draw [->,arc5] (4) ..controls (462.3bp,208.35bp) and (462.06bp,199.03bp)  .. (0);
 
%
\end{tikzpicture}

};
\path 
(brn) edge[->,line width=8pt, color=lightgray] (an)
(an) edge[->,line width=8pt, color=lightgray,bend left] (sg)
;

\end{tikzpicture}
\end{frame}
\end{comment}

\begin{frame}
  \frametitle{Stochastic Automata Networks}
  %\framesubtitle{\tcite{Paulev\'e et al. 2012}}
\begin{columns}
\begin{column}{0.6\textwidth}

% Automate sans les rates
\only<1->{
\tikzstyle{tick}=[densely dotted]
\tikzstyle{hit}=[->,>=angle 45]
\tikzstyle{selfhit}=[min distance=30pt,curve to]
\tikzstyle{bounce}=[densely dotted,>=stealth',->]
\tikzstyle{hlhit}=[very thick]
\begin{center}\scalebox{\scaleex}{
\begin{tikzpicture}
\exandef
\end{tikzpicture}
}\end{center}
}
\end{column}

\begin{column}{0.4\textwidth}
\begin{figure}[p]
\centering

\scalebox{0.4}{

\only<1->{
\tikzstyle{arc0}=[->]
\tikzstyle{nd0}=[]
\tikzstyle{arc1}=[->]
\tikzstyle{nd1}=[]
\tikzstyle{arc2}=[->]
\tikzstyle{nd2}=[]
\tikzstyle{arc3}=[->]
\tikzstyle{nd3}=[]
\tikzstyle{arc4}=[->]
\tikzstyle{nd4}=[]
\tikzstyle{arc5}=[->]
\tikzstyle{nd5}=[]

\begin{tikzpicture}[line join=bevel,font=\LARGE]
%%
  \node (6) at (405.0bp,18.0bp) [reach,nd4] {$\langle a_2,b_1,c_0\rangle$};
  \node (2) at (405.0bp,90.0bp) [reach,nd3] {$\langle a_2,b_0,c_0\rangle$};
  \node (1) at (570.0bp,90.0bp) [reach,nd1] {$\langle a_1,b_0,c_0\rangle$};
  \node (0) at (469.0bp,162.0bp) [reach,nd0] {$s_0 = \langle a_0,b_0,c_0\rangle$};
  \node (4) at (469.0bp,234.0bp) [reach,nd5] {$\langle a_0,b_1,c_0\rangle$};
 
 
  \draw [->,arc3] (0) ..controls (445.65bp,135.46bp) and (435.95bp,124.86bp)  .. (2);
  \draw [->,arc5] (0) ..controls (475.71bp,188.03bp) and (475.94bp,197.36bp)  .. (4);
  \draw [->,arc4] (2) ..controls (405.0bp,63.983bp) and (405.0bp,54.712bp)  .. (6);
  \draw [->,arc1] (0) ..controls (499.93bp,134.79bp) and (517.41bp,122.58bp)  .. (1);
  \draw [->,arc2] (1) ..controls (539.0bp,117.26bp) and (521.52bp,129.47bp)  .. (0);
  \draw [->,arc5] (4) ..controls (462.3bp,208.35bp) and (462.06bp,199.03bp)  .. (0);
 
%
\end{tikzpicture}


}
}

\end{figure}

\end{column}
\end{columns}

\begin{liste}
  \item \tval{Automata}: components \qex{$a$, $b$, $c$}
  \item \tval{local states}: levels of expression \qex{$c_0$, $c_1$, $c_2$}
  \item \tval{States}: sets of active local states%
  \only<1->{\qex{$\PHetat{a_2, b_1, c_0}$}}
  \item \tval{Transitions}: dynamics \qex{\only<1->{$t_1 = \trans{a_0}{a_1}{b_0}$}, \only<1->{$t_2 = \trans{a_1}{a_0}{}$}, \only<1->{$t_3 = \trans{a_0}{a_2}{b_0,c_0}$}, \only<1->{$t_4 = \trans{b_0}{b_1}{}$}}
\end{liste}
\end{frame}

\begin{frame}
  \frametitle{Stochastic Automata Networks}
  %\framesubtitle{\tcite{Paulev\'e et al. 2012}}
\begin{columns}
\begin{column}{0.6\textwidth}

% Automate avec les rates
\only<1->{
\tikzstyle{tick}=[densely dotted]
\tikzstyle{hit}=[->,>=angle 45]
\tikzstyle{selfhit}=[min distance=30pt,curve to]
\tikzstyle{bounce}=[densely dotted,>=stealth',->]
\tikzstyle{hlhit}=[very thick]
\begin{center}\scalebox{\scaleex}{
\begin{tikzpicture}
\exsandef
\end{tikzpicture}
}\end{center}
}
\end{column}

\begin{column}{0.4\textwidth}
\begin{figure}[p]
\centering

\scalebox{0.4}{

\only<1->{
\tikzstyle{arc0}=[->]
\tikzstyle{nd0}=[]
\tikzstyle{arc1}=[->]
\tikzstyle{nd1}=[]
\tikzstyle{arc2}=[->]
\tikzstyle{nd2}=[]
\tikzstyle{arc3}=[->]
\tikzstyle{nd3}=[]
\tikzstyle{arc4}=[->]
\tikzstyle{nd4}=[]
\tikzstyle{arc5}=[->]
\tikzstyle{nd5}=[]

\begin{tikzpicture}[line join=bevel,font=\LARGE]
%%
  \node (6) at (405.0bp,18.0bp) [reach,nd4] {$\langle a_2,b_1,c_0\rangle$};
  \node (2) at (405.0bp,90.0bp) [reach,nd3] {$\langle a_2,b_0,c_0\rangle$};
  \node (1) at (570.0bp,90.0bp) [reach,nd1] {$\langle a_1,b_0,c_0\rangle$};
  \node (0) at (469.0bp,162.0bp) [reach,nd0] {$s_0 = \langle a_0,b_0,c_0\rangle$};
  \node (4) at (469.0bp,234.0bp) [reach,nd5] {$\langle a_0,b_1,c_0\rangle$};
 
 
  \draw [->,arc3] (0) ..controls (445.65bp,135.46bp) and (435.95bp,124.86bp)  .. node[auto] {$\mathbf{\color{Maroon} 2}$}(2) ;
  \draw [->,arc5] (0) ..controls (475.71bp,188.03bp) and (475.94bp,197.36bp)  .. node[right] {$\mathbf{\color{Maroon} 3}$}(4) ;
  \draw [->,arc4] (2) ..controls (405.0bp,63.983bp) and (405.0bp,54.712bp)  .. node[auto] {$\mathbf{\color{Maroon} 3}$}(6) ;
  \draw [->,arc1] (0) ..controls (499.93bp,134.79bp) and (517.41bp,122.58bp)  .. node[auto] {$\mathbf{\color{Maroon} 1}$}(1) ;
  \draw [->,arc2] (1) ..controls (539.0bp,117.26bp) and (521.52bp,129.47bp)  .. node[auto] {$\mathbf{\color{Maroon} 2}$}(0) ;
  \draw [->,arc5] (4) ..controls (462.3bp,208.35bp) and (462.06bp,199.03bp)  .. node[left] {$\mathbf{\color{Maroon} 1}$}(0) ;
 
%
\end{tikzpicture}


}
}

\end{figure}

\end{column}
\end{columns}

\begin{liste}
  \item \tval{Automata}: components \qex{$a$, $b$, $c$}
  \item \tval{local states}: levels of expression \qex{$c_0$, $c_1$, $c_2$}
  \item \tval{States}: sets of active local states%
  \only<1->{\qex{$\PHetat{a_2, b_1, c_0}$}}
  \item \tval{Transitions}: dynamics \qex{\only<1->{$t_1 = \trans{a_0}{a_1}{b_0,\mathbf{\color{Maroon} 2}}$}, \only<1->{$t_2 = \trans{a_1}{a_0}{\mathbf{\color{Maroon} 1}}$}, \only<1->{$t_3 = \trans{a_0}{a_2}{b_0,c_0,\mathbf{\color{Maroon} 2}}$}, \only<1->{$t_4 = \trans{b_0}{b_1}{\mathbf{\color{Maroon} 3}}$}}
\end{liste}
\end{frame}

\begin{frame}
\frametitle{SAN as CTMC}
\begin{figure}[p]
\centering
\scalebox{0.4}{

\begin{tikzpicture}[line join=bevel,font=\LARGE]
%%
  \node (6) at (405.0bp,18.0bp) [reach] {$\langle \mathbf{\color{blue}a_2},b_1,c_0\rangle$};
  \node (2) at (405.0bp,90.0bp) [reach] {$\langle \mathbf{\color{blue}a_2},b_0,c_0\rangle$};

\node (1) at (570.0bp,90.0bp) [reach] {$\langle a_1,b_0,c_0\rangle$};
  \node (64) at (304.0bp,234.0bp) [reach] {$\langle a_0,b_0,c_1\rangle$};
  \node (0) at (469.0bp,162.0bp) [reach] {$\langle a_0,b_0,c_0\rangle}$};
  \node (5) at (507.0bp,306.0bp) [reach] {$\langle a_1,b_1,c_0\rangle$};
  \node (4) at (469.0bp,234.0bp) [reach] {$\langle a_0,b_1,c_0\rangle$};
  \node (69) at (425.0bp,378.0bp) [reach] {$\mathbf{\color{Maroon}s =\langle a_1,b_1,c_1\rangle$};
  \node (68) at (342.0bp,306.0bp) [reach] {$\langle a_0,b_1,c_1\rangle$};
  \node (65) at (232.0bp,450.0bp) [reach] {$\langle a_1,b_0,c_1\rangle$};

  \node[elipse,fill=gray!30] (129) at (88.0bp,90.0bp)  {$\langle a_1,b_0,c_2\rangle$};
  \node[elipse,fill=gray!30] (128) at (139.0bp,162.0bp)  {$\langle a_0,b_0,c_2\rangle$};
  \node[elipse,fill=gray!30] (133) at (112.0bp,306.0bp)  {$\langle a_1,b_1,c_2\rangle$};
  \node[elipse,fill=gray!30] (132) at (139.0bp,234.0bp)  {$\langle a_0,b_1,c_2\rangle$};

  \draw [->] (0) ..controls (445.65bp,135.46bp) and (435.95bp,124.86bp)  .. node[auto] {$\mathbf{\color{Maroon} 2}$}(2);
  \draw [->] (133) ..controls (121.57bp,280.18bp) and (125.26bp,270.62bp)  .. node[auto] {$\mathbf{\color{Maroon} 1}$}(132);
  \draw [->] (65) ..controls (104.12bp,424.19bp) and (0.0bp,387.99bp)  ..  (0.0bp,307.0bp) .. controls (0.0bp,307.0bp) and (0.0bp,307.0bp)  ..  node[auto] {$\mathbf{\color{Maroon} 2}$} (0.0bp,233.0bp) .. controls (0.0bp,185.75bp) and (35.64bp,141.13bp)  .. node[auto] {$\mathbf{\color{Maroon} 3}$}(129);
  \draw [->] (65) ..controls (223.97bp,401.24bp) and (228.52bp,336.68bp)  .. (251.0bp,288.0bp) .. controls (255.96bp,277.25bp) and (263.67bp,266.89bp)  .. node[auto] {$\mathbf{\color{Maroon} 1}$}(64);
  \draw [->] (5) ..controls (483.32bp,333.07bp) and (470.02bp,344.55bp)  .. node[auto] {$\mathbf{\color{Maroon} 2}$}(69);
  \draw [->] (64) ..controls (244.15bp,207.61bp) and (210.57bp,193.36bp)  .. node[auto] {$\mathbf{\color{Maroon} 2}$} (128);
  \draw [->] (5) ..controls (493.43bp,280.01bp) and (488.11bp,270.2bp)  .. node[auto] {$\mathbf{\color{Maroon} 1}$}(4);
  \draw [->] (0) ..controls (475.71bp,188.03bp) and (475.94bp,197.36bp)  .. node[right] {$\mathbf{\color{Maroon} 3}$}(4);
  \draw [->] (1) ..controls (588.79bp,134.38bp) and (608.0bp,186.55bp)  .. (608.0bp,233.0bp) .. controls (608.0bp,307.0bp) and (608.0bp,307.0bp)  .. (608.0bp,307.0bp) .. controls (608.0bp,366.83bp) and (560.73bp,369.68bp)  .. (507.0bp,396.0bp) .. controls (446.14bp,425.81bp) and (370.03bp,438.87bp)  .. node[auto] {$\mathbf{\color{Maroon} 3}$}(65);
  \draw [->] (1) ..controls (572.33bp,138.66bp) and (571.97bp,203.11bp)  .. (551.0bp,252.0bp) .. controls (546.57bp,262.33bp) and (539.46bp,272.21bp)  .. node[right] {$\mathbf{\color{Maroon} 3}$}(5);
  \draw [->] (129) ..controls (68.077bp,117.99bp) and (60.513bp,131.04bp)  .. (57.0bp,144.0bp) .. controls (44.446bp,190.33bp) and (38.221bp,207.83bp)  .. (57.0bp,252.0bp) .. controls (61.945bp,263.63bp) and (70.831bp,273.95bp)  .. node[auto] {$\mathbf{\color{Maroon} 3}$}(133);
  \draw [->] (128) ..controls (145.71bp,188.03bp) and (145.94bp,197.36bp)  .. node[right] {$\mathbf{\color{Maroon} 3}$}(132);
  \draw [->] (69) ..controls (448.8bp,350.83bp) and (462.16bp,339.29bp)  .. node[auto] {$\mathbf{\color{Maroon} 3}$}(5);
  \draw [->] (2) ..controls (405.0bp,63.983bp) and (405.0bp,54.712bp)  .. node[auto] {$\mathbf{\color{Maroon} 3}$}(6);
  \draw [->] (0) ..controls (499.93bp,134.79bp) and (517.41bp,122.58bp)  .. node[auto] {$\mathbf{\color{Maroon} 1}$}(1);
  \draw [->] (68) ..controls (388.56bp,279.34bp) and (412.1bp,266.36bp)  .. node[auto] {$\mathbf{\color{Maroon} 1}$}(4);
  \draw [->] (68) ..controls (321.75bp,280.09bp) and (316.27bp,270.41bp)  .. node[auto] {$\mathbf{\color{Maroon} 3}$}(64);
  \draw [->] (1) ..controls (539.0bp,117.26bp) and (521.52bp,129.47bp)  .. node[auto] {$\mathbf{\color{Maroon} 2}$}(0);
  \draw [->] (65) ..controls (301.71bp,423.72bp) and (343.47bp,408.57bp)  .. node[auto] {$\mathbf{\color{Maroon} 3}$}(69);
  \draw [->] (64) ..controls (324.32bp,260.04bp) and (329.77bp,269.67bp)  .. node[auto] {$\mathbf{\color{Maroon} 1}$}(68);
  \draw [->] (69) ..controls (394.58bp,351.34bp) and (381.1bp,339.97bp)  .. node[auto] {$\mathbf{\color{Maroon} 1}$}(68);
  \draw [->] (128) ..controls (114.09bp,136.08bp) and (106.38bp,125.8bp)  .. node[auto] {$\mathbf{\color{Maroon} 1}$}(129);
  \draw [->] (64) ..controls (285.73bp,261.68bp) and (275.22bp,274.54bp)  .. (269.0bp,288.0bp) .. controls (248.81bp,331.74bp) and (243.08bp,388.29bp)  .. node[auto] {$\mathbf{\color{Maroon} 2}$}(65);
  \draw [->] (132) ..controls (132.3bp,208.35bp) and (132.06bp,199.03bp)  .. node[auto] {$\mathbf{\color{Maroon} 1}$}(128);
  \draw [->] (4) ..controls (462.3bp,208.35bp) and (462.06bp,199.03bp)  .. node[left] {$\mathbf{\color{Maroon} 1}$}(0);
  \draw [->] (129) ..controls (113.05bp,116.11bp) and (120.71bp,126.32bp)  .. node[auto] {$\mathbf{\color{Maroon} 2}$}(128);
%
\end{tikzpicture}


}

\label{ex-bifurcations}
\end{figure}
Bad news: \textcolor{red}{state space explosion}
\end{frame}

\begin{frame}
\frametitle{SAN as CTMC}
\begin{columns}
 \begin{column}{0.5\textwidth}
 \begin{figure}[p]
\centering
\scalebox{0.25}{

\begin{tikzpicture}[line join=bevel,font=\LARGE]
%%
  \node (6) at (405.0bp,18.0bp) [reach] {$\langle \mathbf{\color{blue}a_2},b_1,c_0\rangle$};
  \node (2) at (405.0bp,90.0bp) [reach] {$\langle \mathbf{\color{blue}a_2},b_0,c_0\rangle$};

\node (1) at (570.0bp,90.0bp) [reach] {$\langle a_1,b_0,c_0\rangle$};
  \node (64) at (304.0bp,234.0bp) [reach] {$\langle a_0,b_0,c_1\rangle$};
  \node (0) at (469.0bp,162.0bp) [reach] {$\langle a_0,b_0,c_0\rangle}$};
  \node (5) at (507.0bp,306.0bp) [reach] {$\langle a_1,b_1,c_0\rangle$};
  \node (4) at (469.0bp,234.0bp) [reach] {$\langle a_0,b_1,c_0\rangle$};
  \node (69) at (425.0bp,378.0bp) [reach] {$\mathbf{\color{Maroon}s =\langle a_1,b_1,c_1\rangle$};
  \node (68) at (342.0bp,306.0bp) [reach] {$\langle a_0,b_1,c_1\rangle$};
  \node (65) at (232.0bp,450.0bp) [reach] {$\langle a_1,b_0,c_1\rangle$};

  \node[elipse,fill=gray!30] (129) at (88.0bp,90.0bp)  {$\langle a_1,b_0,c_2\rangle$};
  \node[elipse,fill=gray!30] (128) at (139.0bp,162.0bp)  {$\langle a_0,b_0,c_2\rangle$};
  \node[elipse,fill=gray!30] (133) at (112.0bp,306.0bp)  {$\langle a_1,b_1,c_2\rangle$};
  \node[elipse,fill=gray!30] (132) at (139.0bp,234.0bp)  {$\langle a_0,b_1,c_2\rangle$};

  \draw [->] (0) ..controls (445.65bp,135.46bp) and (435.95bp,124.86bp)  .. node[auto] {$\mathbf{\color{Maroon} 2}$}(2);
  \draw [->] (133) ..controls (121.57bp,280.18bp) and (125.26bp,270.62bp)  .. node[auto] {$\mathbf{\color{Maroon} 1}$}(132);
  \draw [->] (65) ..controls (104.12bp,424.19bp) and (0.0bp,387.99bp)  ..  (0.0bp,307.0bp) .. controls (0.0bp,307.0bp) and (0.0bp,307.0bp)  ..  node[auto] {$\mathbf{\color{Maroon} 2}$} (0.0bp,233.0bp) .. controls (0.0bp,185.75bp) and (35.64bp,141.13bp)  .. node[auto] {$\mathbf{\color{Maroon} 3}$}(129);
  \draw [->] (65) ..controls (223.97bp,401.24bp) and (228.52bp,336.68bp)  .. (251.0bp,288.0bp) .. controls (255.96bp,277.25bp) and (263.67bp,266.89bp)  .. node[auto] {$\mathbf{\color{Maroon} 1}$}(64);
  \draw [->] (5) ..controls (483.32bp,333.07bp) and (470.02bp,344.55bp)  .. node[auto] {$\mathbf{\color{Maroon} 2}$}(69);
  \draw [->] (64) ..controls (244.15bp,207.61bp) and (210.57bp,193.36bp)  .. node[auto] {$\mathbf{\color{Maroon} 2}$} (128);
  \draw [->] (5) ..controls (493.43bp,280.01bp) and (488.11bp,270.2bp)  .. node[auto] {$\mathbf{\color{Maroon} 1}$}(4);
  \draw [->] (0) ..controls (475.71bp,188.03bp) and (475.94bp,197.36bp)  .. node[right] {$\mathbf{\color{Maroon} 3}$}(4);
  \draw [->] (1) ..controls (588.79bp,134.38bp) and (608.0bp,186.55bp)  .. (608.0bp,233.0bp) .. controls (608.0bp,307.0bp) and (608.0bp,307.0bp)  .. (608.0bp,307.0bp) .. controls (608.0bp,366.83bp) and (560.73bp,369.68bp)  .. (507.0bp,396.0bp) .. controls (446.14bp,425.81bp) and (370.03bp,438.87bp)  .. node[auto] {$\mathbf{\color{Maroon} 3}$}(65);
  \draw [->] (1) ..controls (572.33bp,138.66bp) and (571.97bp,203.11bp)  .. (551.0bp,252.0bp) .. controls (546.57bp,262.33bp) and (539.46bp,272.21bp)  .. node[right] {$\mathbf{\color{Maroon} 3}$}(5);
  \draw [->] (129) ..controls (68.077bp,117.99bp) and (60.513bp,131.04bp)  .. (57.0bp,144.0bp) .. controls (44.446bp,190.33bp) and (38.221bp,207.83bp)  .. (57.0bp,252.0bp) .. controls (61.945bp,263.63bp) and (70.831bp,273.95bp)  .. node[auto] {$\mathbf{\color{Maroon} 3}$}(133);
  \draw [->] (128) ..controls (145.71bp,188.03bp) and (145.94bp,197.36bp)  .. node[right] {$\mathbf{\color{Maroon} 3}$}(132);
  \draw [->] (69) ..controls (448.8bp,350.83bp) and (462.16bp,339.29bp)  .. node[auto] {$\mathbf{\color{Maroon} 3}$}(5);
  \draw [->] (2) ..controls (405.0bp,63.983bp) and (405.0bp,54.712bp)  .. node[auto] {$\mathbf{\color{Maroon} 3}$}(6);
  \draw [->] (0) ..controls (499.93bp,134.79bp) and (517.41bp,122.58bp)  .. node[auto] {$\mathbf{\color{Maroon} 1}$}(1);
  \draw [->] (68) ..controls (388.56bp,279.34bp) and (412.1bp,266.36bp)  .. node[auto] {$\mathbf{\color{Maroon} 1}$}(4);
  \draw [->] (68) ..controls (321.75bp,280.09bp) and (316.27bp,270.41bp)  .. node[auto] {$\mathbf{\color{Maroon} 3}$}(64);
  \draw [->] (1) ..controls (539.0bp,117.26bp) and (521.52bp,129.47bp)  .. node[auto] {$\mathbf{\color{Maroon} 2}$}(0);
  \draw [->] (65) ..controls (301.71bp,423.72bp) and (343.47bp,408.57bp)  .. node[auto] {$\mathbf{\color{Maroon} 3}$}(69);
  \draw [->] (64) ..controls (324.32bp,260.04bp) and (329.77bp,269.67bp)  .. node[auto] {$\mathbf{\color{Maroon} 1}$}(68);
  \draw [->] (69) ..controls (394.58bp,351.34bp) and (381.1bp,339.97bp)  .. node[auto] {$\mathbf{\color{Maroon} 1}$}(68);
  \draw [->] (128) ..controls (114.09bp,136.08bp) and (106.38bp,125.8bp)  .. node[auto] {$\mathbf{\color{Maroon} 1}$}(129);
  \draw [->] (64) ..controls (285.73bp,261.68bp) and (275.22bp,274.54bp)  .. (269.0bp,288.0bp) .. controls (248.81bp,331.74bp) and (243.08bp,388.29bp)  .. node[auto] {$\mathbf{\color{Maroon} 2}$}(65);
  \draw [->] (132) ..controls (132.3bp,208.35bp) and (132.06bp,199.03bp)  .. node[auto] {$\mathbf{\color{Maroon} 1}$}(128);
  \draw [->] (4) ..controls (462.3bp,208.35bp) and (462.06bp,199.03bp)  .. node[left] {$\mathbf{\color{Maroon} 1}$}(0);
  \draw [->] (129) ..controls (113.05bp,116.11bp) and (120.71bp,126.32bp)  .. node[auto] {$\mathbf{\color{Maroon} 2}$}(128);
%
\end{tikzpicture}


}

\label{ex-bifurcations}
\end{figure}
Bad news: \textcolor{red}{state space explosion}
  
 \end{column}
 
 \begin{column}{0.5\textwidth}
  \textbf{Probability to have a transition from state s:}
  $$\ppexit{s}{*}{t} = 1-e^{-\lambda t}$$ \\
  \medskip
  \textbf{Probability to have a transition from s to s':}
  $$\ppexit{s}{s^{'}}{t} = \frac{R (s,s^{'})}{E (s)} \cdot \ppexit{s}{*}{t}$$\\
  \medskip
  \textbf{Probability to have a transition from s to a set of states:}
  $$\ppexit{s}{A}{t} = \frac{R (s,A)}{E (s)} \cdot \ppexit{s}{*}{t}$$ \\
 \end{column}

\end{columns}

\end{frame}


