% Exemples de structures abstraites (graphes de causalité locale)

\begin{frame}[t]
  \frametitle{Static analysis of quantitative properties}

\def \tu {3}
\def \tub {4}
\def \tuf {5}

\begin{columns}[t]
\begin{column}{0.48\textwidth}
\begin{center}
\scalebox{0.55}{
\begin{tikzpicture}
\exattnew
\TState{-\tu}{a_0,b_0,bs_0,c_1,d_0,e_1}
\TState{\tub-}{a_1,b_0,bs_1,c_1,d_1,e_1}
\node[process,very thick] (d_2) at (d_2.center) {?};
\end{tikzpicture}
}
\end{center}

\end{column}
\begin{column}{0.52\textwidth}

\uncover<2->{
%\vspace{1em}
\tval{Tools:}

\smallskip
\begin{itemize}
  \item \tval{$CCSL$}(Conditional Continous Stochastic Logic) (Gao et al.,2013)
  \item at each step, define a probability mesure $\Rightarrow $ \tval{Probabilistic Semantics}
\end{itemize}

%il y a de la place ici 
\begin{center}
 The probability is at least $0.1$, that the number of proteins is more than $5$
 and the gene becomes inactive within time interval $[10,20)$, under the condition that the
 proteins increasingly accumulated from $0$ to k within the same time interval $[10,20)$
\tval{$Pr_{\geq 0.1}(\Diamond_{[10,20)} f \wedge  g | f_{1} U_{[10,20)} f_{2} U_{[10,20)} \ldots f_{k} )$}
\end{center}

}

\end{column}
\end{columns}

\begin{center}%
\scalebox{\scaleex}{%
\only<-\tu>{%
\scalebox{\scaleex}{%
\begin{tikzpicture}[aS]
  \path[use as bounding box] (.7,0) rectangle (5.8,3.5);

  \glclegend{}{$d_2$}{$\PHobj{d_0}{d_2}$}
\end{tikzpicture}
}
%~~~\sauyes
~~~\sasaquant
}
\only<\tub->{
  \sasaquant
}}
\end{center}
\end{frame}
