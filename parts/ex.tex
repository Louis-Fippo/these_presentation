% Exemples

%%% Exemple pour la définition du Process Hitting %%%
\def \exphdef {
\path[use as bounding box] (-0.5,-0.5) rectangle (6.5,4.5);

\TSort{(0,3)}{a}{2}{l}
\TSort{(0,0)}{b}{2}{l}
\TSort{(6,1)}{z}{3}{r}

\THit{a_1}{}{z_1}{.west}{z_2}
\THit{b_1}{}{z_0}{.west}{z_1}
\THit{a_0}{out=250,in=200,selfhit}{a_0}{.west}{a_1}

\path[bounce,bend left]
\TBounce{z_0}{}{z_1}{.south}
\TBounce{z_1}{}{z_2}{.south}
\TBounce{a_0}{}{a_1}{.south}
;
}



%%% Exemple pour la coopération %%%
\def \exphcoop {
\path[use as bounding box] (-0.5,-0.5) rectangle (6.5,4.5);

% Actions de màj grisées
\only<6->{
\THit{a_1}{ulhit,color=lightgray}{ab_0}{.west}{ab_2}
\THit{a_1}{ulhit,color=lightgray}{ab_1}{.west}{ab_3}
\path[bounce,bend left,pulhit] \TBounce{ab_0}{bulhit}{ab_2}{.south} \TBounce{ab_1}{bulhit}{ab_3}{.south} ;
}

\only<7->{
\THit{a_0}{ulhit}{ab_2}{.west}{ab_0}
\THit{a_0}{ulhit}{ab_3}{.west}{ab_1}
\path[bounce,bend right,pulhit] \TBounce{ab_2}{bulhit}{ab_0}{.north} \TBounce{ab_3}{bulhit}{ab_1}{.north} ;
}

\only<8->{
\THit{b_0}{ulhit}{ab_3}{.west}{ab_2}
\THit{b_0}{ulhit}{ab_1}{.west}{ab_0}
\THit{b_1}{ulhit}{ab_0}{.west}{ab_1}
\THit{b_1}{ulhit}{ab_2}{.west}{ab_3}
\path[bounce,bend right,pulhit] \TBounce{ab_1}{bulhit}{ab_0}{.north} \TBounce{ab_3}{bulhit}{ab_2}{.north} ;
\path[bounce,bend left,pulhit] \TBounce{ab_0}{bulhit}{ab_1}{.south} \TBounce{ab_2}{bulhit}{ab_3}{.south} ;
}

% Sortes
\TSort{(0,3)}{a}{2}{l}
\TSort{(0,0)}{b}{2}{l}
\TSort{(6,1)}{z}{3}{r}

% Deux actions disjointes en exemple
\only<2-3>{
\THit{a_1}{}{z_1}{.north west}{z_2}
\path[bounce,bend left]
\TBounce{z_1}{}{z_2}{.south} ;

\THit{b_0}{}{z_1}{.west}{z_2}
\path[bounce,bend left=55]
\TBounce{z_1}{}{z_2}{.south west} ;
}

% Processus d'exemple
\TState{3}{a_1,b_1,z_1}

% Sorte coopérative et arcs
\only<4->{
\TSetTick{ab}{0}{00}
\TSetTick{ab}{1}{01}
\TSetTick{ab}{2}{10}
\TSetTick{ab}{3}{11}
\TSort{(3,0.5)}{ab}{4}{l}
}

% Arcs de màj noirs de la sc
\only<5>{
\THit{a_1}{thick}{ab_0}{.west}{ab_2}
\THit{a_1}{thick}{ab_1}{.west}{ab_3}
\path[bounce,thick,bend left] \TBounce{ab_0}{thick}{ab_2}{.south} \TBounce{ab_1}{thick}{ab_3}{.south} ;
}

\only<6>{
\THit{a_0}{thick}{ab_2}{.west}{ab_0}
\THit{a_0}{thick}{ab_3}{.west}{ab_1}
\path[bounce,thick,bend right] \TBounce{ab_2}{thick}{ab_0}{.north} \TBounce{ab_3}{thick}{ab_1}{.north} ;
}

\only<7>{
\THit{b_0}{thick}{ab_3}{.west}{ab_2}
\THit{b_0}{thick}{ab_1}{.west}{ab_0}
\THit{b_1}{thick}{ab_0}{.west}{ab_1}
\THit{b_1}{thick}{ab_2}{.west}{ab_3}
\path[bounce,thick,bend right] \TBounce{ab_1}{thick}{ab_0}{.north} \TBounce{ab_3}{thick}{ab_2}{.north} ;
\path[bounce,thick,bend left] \TBounce{ab_0}{thick}{ab_1}{.south} \TBounce{ab_2}{thick}{ab_3}{.south} ;
}

% État d'exemple pour màj de la sc
\TState{8-9}{a_1,b_0}
\TState{10}{a_1,b_0,ab_0,ab_1,ab_2,ab_3}
\TState{11}{a_1,b_0,ab_2}
\only<9-11>{
\THit{a_1}{}{ab_0}{.west}{ab_2}
\THit{a_1}{}{ab_1}{.west}{ab_3}
\THit{b_0}{}{ab_3}{.west}{ab_2}
\THit{b_0}{}{ab_1}{.west}{ab_0}
\path[bounce,bend left] \TBounce{ab_0}{}{ab_2}{.south} \TBounce{ab_1}{}{ab_3}{.south} ;
\path[bounce,bend right] \TBounce{ab_1}{}{ab_0}{.north} \TBounce{ab_3}{}{ab_2}{.north} ;
}

% État d'exemple pour action de la sc
\TState{12}{a_1,b_0,z_1,ab_2}
\TState{13-14}{a_1,b_0,z_2,ab_2}

% Arc sortant de la sc
\only<12-14>{
\THit{ab_2}{thick}{z_1}{.west}{z_2}
\path[bounce,bend left,thick] \TBounce{z_1}{thick}{z_2}{.south} ;
}

% Arc sortant de la sc
\only<15->{
\THit{ab_2}{}{z_1}{.west}{z_2}
\path[bounce,bend left] \TBounce{z_1}{}{z_2}{.south} ;
}

}



%%% Exemple pour l'inférence %%%
\def \exphinf {
% Sortes
\TSort{(0,3)}{a}{2}{l}
\TSort{(0,0)}{b}{2}{l}
\TSort{(6,0)}{z}{3}{r}

% Sorte coopérative et arcs
\TSetTick{ab}{0}{00}
\TSetTick{ab}{1}{01}
\TSetTick{ab}{2}{10}
\TSetTick{ab}{3}{11}
\TSort{(3,0)}{ab}{4}{l}

% Actions de màj grisées
\THit{a_1}{ulhit}{ab_0}{.west}{ab_2}
\THit{a_1}{ulhit}{ab_1}{.west}{ab_3}
\path[bounce,bend left,pulhit] \TBounce{ab_0}{bulhit}{ab_2}{.south} \TBounce{ab_1}{bulhit}{ab_3}{.south};

\THit{a_0}{ulhit}{ab_2}{.west}{ab_0}
\THit{a_0}{ulhit}{ab_3}{.west}{ab_1}
\path[bounce,bend right,pulhit] \TBounce{ab_2}{bulhit}{ab_0}{.north} \TBounce{ab_3}{bulhit}{ab_1}{.north};

\THit{b_0}{ulhit}{ab_3}{.west}{ab_2}
\THit{b_0}{ulhit}{ab_1}{.west}{ab_0}
\THit{b_1}{ulhit}{ab_0}{.west}{ab_1}
\THit{b_1}{ulhit}{ab_2}{.west}{ab_3}
\path[bounce,bend right,pulhit] \TBounce{ab_1}{bulhit}{ab_0}{.north} \TBounce{ab_3}{bulhit}{ab_2}{.north};
\path[bounce,bend left,pulhit] \TBounce{ab_0}{bulhit}{ab_1}{.south} \TBounce{ab_2}{bulhit}{ab_3}{.south};

% Arcs sortant de la sc
\THit{ab_2}{ulhit}{z_1}{.north west}{z_2}
\THit{ab_2}{ulhit}{z_0}{.west}{z_1}
\path[bounce,bend left,pulhit] \TBounce{z_1}{bulhit}{z_2}{.south} \TBounce{z_0}{bulhit}{z_1}{.south};

\THit{ab_3}{ulhit}{z_2}{.west}{z_1}
\THit{ab_3}{ulhit}{z_0}{.west}{z_1}
\THit{ab_1}{ulhit}{z_2}{.west}{z_1}
\THit{ab_1}{ulhit}{z_0}{.west}{z_1}
\path[bounce,bend left,pulhit] \TBounce{z_2}{bulhit,bend right}{z_1}{.north};

\THit{ab_0}{ulhit}{z_2}{.west}{z_1}
\THit{ab_0}{ulhit}{z_1}{.south west}{z_0}
\path[bounce,bend right,pulhit] \TBounce{z_2}{bulhit}{z_1}{.north} \TBounce{z_1}{bulhit}{z_0}{.north};

}



%%% Exemple pour l'inférence (sans arcs grisés) %%%
\def \exphinfblack {
% Sortes
\TSort{(0,3)}{a}{2}{l}
\TSort{(0,0)}{b}{2}{l}
\TSort{(6,0)}{z}{3}{r}

% Sorte coopérative et arcs
\TSetTick{ab}{0}{00}
\TSetTick{ab}{1}{01}
\TSetTick{ab}{2}{10}
\TSetTick{ab}{3}{11}
\TSort{(3,0)}{ab}{4}{l}

% Actions de màj grisées
\THit{a_1}{}{ab_0}{.west}{ab_2}
\THit{a_1}{}{ab_1}{.west}{ab_3}
\path[bounce,bend left] \TBounce{ab_0}{}{ab_2}{.south} \TBounce{ab_1}{}{ab_3}{.south};

\THit{a_0}{}{ab_2}{.west}{ab_0}
\THit{a_0}{}{ab_3}{.west}{ab_1}
\path[bounce,bend right] \TBounce{ab_2}{}{ab_0}{.north} \TBounce{ab_3}{}{ab_1}{.north};

\THit{b_0}{}{ab_3}{.west}{ab_2}
\THit{b_0}{}{ab_1}{.west}{ab_0}
\THit{b_1}{}{ab_0}{.west}{ab_1}
\THit{b_1}{}{ab_2}{.west}{ab_3}
\path[bounce,bend right] \TBounce{ab_1}{}{ab_0}{.north} \TBounce{ab_3}{}{ab_2}{.north};
\path[bounce,bend left] \TBounce{ab_0}{}{ab_1}{.south} \TBounce{ab_2}{}{ab_3}{.south};

% Arcs sortant de la sc
\THit{ab_2}{}{z_1}{.north west}{z_2}
\THit{ab_2}{}{z_0}{.west}{z_1}
\path[bounce,bend left] \TBounce{z_1}{}{z_2}{.south} \TBounce{z_0}{}{z_1}{.south};

\THit{ab_3}{}{z_2}{.west}{z_1}
\THit{ab_3}{}{z_0}{.west}{z_1}
\THit{ab_1}{}{z_2}{.west}{z_1}
\THit{ab_1}{}{z_0}{.west}{z_1}
\path[bounce,bend left] \TBounce{z_2}{,bend right}{z_1}{.north};

\THit{ab_0}{}{z_2}{.west}{z_1}
\THit{ab_0}{}{z_1}{.south west}{z_0}
\path[bounce,bend right] \TBounce{z_2}{}{z_1}{.north} \TBounce{z_1}{}{z_0}{.north};

}



%%% Exemple 2 pour l'inférence (projections) %%%
\def \exphinfproj {
% Sortes
\TSort{(0,3)}{a}{2}{l}
\TSort{(0,0)}{b}{2}{l}
\TSort{(6,1)}{z}{2}{r}

% Sorte coopérative et arcs
\TSetTick{ab}{0}{00}
\TSetTick{ab}{1}{01}
\TSetTick{ab}{2}{10}
\TSetTick{ab}{3}{11}
\TSort{(3,0)}{ab}{4}{l}

% Actions de màj grisées
\THit{a_1}{ulhit}{ab_0}{.west}{ab_2}
\THit{a_1}{ulhit}{ab_1}{.west}{ab_3}
\path[bounce,bend left,pulhit] \TBounce{ab_0}{bulhit}{ab_2}{.south} \TBounce{ab_1}{bulhit}{ab_3}{.south} ;

\THit{a_0}{ulhit}{ab_2}{.west}{ab_0}
\THit{a_0}{ulhit}{ab_3}{.west}{ab_1}
\path[bounce,bend right,pulhit] \TBounce{ab_2}{bulhit}{ab_0}{.north} \TBounce{ab_3}{bulhit}{ab_1}{.north} ;

\THit{b_0}{ulhit}{ab_3}{.west}{ab_2}
\THit{b_0}{ulhit}{ab_1}{.west}{ab_0}
\THit{b_1}{ulhit}{ab_0}{.west}{ab_1}
\THit{b_1}{ulhit}{ab_2}{.west}{ab_3}
\path[bounce,bend right,pulhit] \TBounce{ab_1}{bulhit}{ab_0}{.north} \TBounce{ab_3}{bulhit}{ab_2}{.north} ;
\path[bounce,bend left,pulhit] \TBounce{ab_0}{bulhit}{ab_1}{.south} \TBounce{ab_2}{bulhit}{ab_3}{.south} ;

% Arcs sortant de la sc
\THit{ab_3}{ulhit}{z_0}{.west}{z_1}
\path[bounce,bend left,pulhit] \TBounce{z_0}{bulhit}{z_1}{.south} ;

\THit{ab_0}{ulhit}{z_1}{.west}{z_0}
\path[bounce,bend right,pulhit]\TBounce{z_1}{bulhit}{z_0}{.north} ;
}



%%% Exemple sans sorte coopérative pour l'inférence %%%
\def \exphinfprojssc {
% Sortes
\TSort{(0,3)}{a}{2}{l}
\TSort{(0,0)}{b}{2}{l}
\TSort{(6,0)}{z}{3}{r}

\THit{a_1}{ulhit}{z_0}{.west}{z_1}
\THit{a_1}{ulhit}{z_1}{.north west}{z_2}
\THit{a_0}{ulhit}{z_1}{.south west}{z_0}
\THit{a_0}{ulhit}{z_2}{.west}{z_1}
\path[bounce,bend left,pulhit] \TBounce{z_0}{bulhit}{z_1}{.south} \TBounce{z_1}{bulhit}{z_2}{.south}
  \TBounce{z_1}{bulhit,bend right}{z_0}{.north} \TBounce{z_2}{bulhit,bend right}{z_1}{.north} ;

\THit{b_0}{ulhit}{z_0}{.west}{z_1}
\THit{b_0}{ulhit}{z_1}{.north west}{z_2}
\THit{b_1}{ulhit}{z_1}{.south west}{z_0}
\THit{b_1}{ulhit}{z_2}{.west}{z_1}
%\path[bounce,bend left,pulhit] \TBounce{z_0}{bulhit}{z_1}{.south} \TBounce{z_1}{bulhit}{z_2}{.south}
%  \TBounce{z_1}{bulhit,bend right}{z_0}{.north} \TBounce{z_2}{bulhit,bend right}{z_1}{.north} ;
}

%%%%exemple pour la transformation en pattern  activation%%%%
\def \exphpatact {
\path[use as bounding box] (-0.5,-0.5) rectangle (2.5,2.5);

\TSort{(0,0.5)}{a}{2}{l}
\TSort{(2,0.5)}{b}{2}{l}
%\TSort{(6,1)}{z}{3}{r}

\THit{a_1}{}{b_0}{.west}{b_1}
\THit{a_0}{}{b_1}{.west}{b_0}
%\THit{a_0}{out=250,in=200,selfhit}{a_0}{.west}{a_1}

\path[bounce,bend left]
\TBounce{b_0}{}{b_1}{.south}
\TBounce{b_1}{bend right}{b_0}{.north}
%\TBounce{a_0}{}{a_1}{.south}
;
}

%%%%exemple pour la transformation en pattern inibition%%%%
\def \exphpatini {
\path[use as bounding box] (-0.5,-0.5) rectangle (2.5,2.5);

\TSort{(0,0.5)}{a}{2}{l}
\TSort{(2,0.5)}{b}{2}{l}
%\TSort{(6,1)}{z}{3}{r}

\THit{a_1}{}{b_1}{.west}{b_0}
\THit{a_0}{}{b_0}{.west}{b_1}
%\THit{a_0}{out=250,in=200,selfhit}{a_0}{.west}{a_1}

\path[bounce,bend left]
\TBounce{b_1}{bend right}{b_0}{.north}
\TBounce{b_0}{}{b_1}{.south}
%\TBounce{a_0}{}{a_1}{.south}
;
}

%%%%exemple pour la transformation en pattern c est soit activé par a ou inibé par b%%%%
\def \exphpatai {
\path[use as bounding box] (-0.5,-0.5) rectangle (2.5,5.5);

\TSort{(0,0)}{a}{2}{l}
\TSort{(0,3)}{b}{2}{l}
\TSort{(2,1)}{c}{2}{r}

\THit{a_1}{}{c_0}{.west}{c_1}
\THit{a_0}{}{c_1}{.west}{c_0}
\THit{b_1}{}{c_1}{.west}{c_0}
\THit{b_0}{}{c_0}{.west}{c_1}

%\THit{a_0}{out=250,in=200,selfhit}{a_0}{.west}{a_1}

\path[bounce,bend left]
\TBounce{c_1}{bend right}{c_0}{.north}
\TBounce{c_0}{}{c_1}{.south}
%\TBounce{a_0}{}{a_1}{.south}
;
}

%%%%exemple pour la transformation en pattern c est soit activé par a ou activé par b%%%%
\def \exphpataa {
\path[use as bounding box] (-0.5,-0.5) rectangle (2.5,5.5);

\TSort{(0,0)}{a}{2}{l}
\TSort{(0,3)}{b}{2}{l}
\TSort{(2,1)}{c}{2}{r}

\THit{a_1}{}{c_0}{.west}{c_1}
\THit{a_0}{}{c_1}{.west}{c_0}
\THit{b_1}{}{c_0}{.west}{c_1}
\THit{b_0}{}{c_1}{.west}{c_0}

%\THit{a_0}{out=250,in=200,selfhit}{a_0}{.west}{a_1}

\path[bounce,bend left]
\TBounce{c_1}{bend right}{c_0}{.north}
\TBounce{c_0}{}{c_1}{.south}
%\TBounce{a_0}{}{a_1}{.south}
;
}

%%%%exemple pour la transformation en pattern c est soit activé par a ou activé par b%%%%
\def \exphpataar {
\path[use as bounding box] (-0.5,-0.5) rectangle (2.5,5.5);

\TSort{(0,0)}{a}{2}{l}
\TSort{(0,3)}{b}{2}{l}
\TSort{(2,1)}{syn}{4}{r}
\TSort{(4,1)}{c}{2}{r}

\THit{a_1}{}{c_0}{.north west}{c_1}
%\THit{a_0}{}{c_1}{.west}{c_0}
%\THit{a_0}{}{syn_1}{.west}{c_0}
\THit{b_1}{}{c_0}{.north west}{c_1}
%\THit{b_0}{}{c_1}{.west}{c_0}
%\THit{b_0}{}{syn_1}{.west}{c_0}
\THit{syn_0}{}{c_1}{.west}{c_0}

%\THit{a_0}{out=250,in=200,selfhit}{a_0}{.west}{a_1}

\path[bounce,bend left]
\TBounce{c_1}{bend right}{c_0}{.north}
\TBounce{c_0}{}{c_1}{.south}
%\TBounce{a_0}{}{a_1}{.south}
;
}

%%% Exemple pour la coopération %%%
\def \exphsyn {
\path[use as bounding box] (-0.5,-0.5) rectangle (6.5,4.5);

% Actions de màj grisées
\only<6->{
\THit{a_1}{ulhit,color=lightgray}{ab_0}{.west}{ab_2}
\THit{a_1}{ulhit,color=lightgray}{ab_1}{.west}{ab_3}
\path[bounce,bend left,pulhit] \TBounce{ab_0}{bulhit}{ab_2}{.south} \TBounce{ab_1}{bulhit}{ab_3}{.south} ;
}

\only<7->{
\THit{a_0}{ulhit}{ab_2}{.west}{ab_0}
\THit{a_0}{ulhit}{ab_3}{.west}{ab_1}
\path[bounce,bend right,pulhit] \TBounce{ab_2}{bulhit}{ab_0}{.north} \TBounce{ab_3}{bulhit}{ab_1}{.north} ;
}

\only<8->{
\THit{b_0}{ulhit}{ab_3}{.west}{ab_2}
\THit{b_0}{ulhit}{ab_1}{.west}{ab_0}
\THit{b_1}{ulhit}{ab_0}{.west}{ab_1}
\THit{b_1}{ulhit}{ab_2}{.west}{ab_3}
\path[bounce,bend right,pulhit] \TBounce{ab_1}{bulhit}{ab_0}{.north} \TBounce{ab_3}{bulhit}{ab_2}{.north} ;
\path[bounce,bend left,pulhit] \TBounce{ab_0}{bulhit}{ab_1}{.south} \TBounce{ab_2}{bulhit}{ab_3}{.south} ;
}

% Sortes
\TSort{(0,3)}{a}{2}{l}
\TSort{(0,0)}{b}{2}{l}
\TSort{(6,1)}{z}{3}{r}

% Deux actions disjointes en exemple
\only<2-3>{
\THit{a_1}{}{z_1}{.north west}{z_2}
\path[bounce,bend left]
\TBounce{z_1}{}{z_2}{.south} ;

\THit{b_1}{}{z_1}{.west}{z_2}
\path[bounce,bend left=55]
\TBounce{z_1}{}{z_2}{.south west} ;
}

% Processus d'exemple
\TState{3}{a_1,b_1,z_1}

% Sorte coopérative et arcs
\only<4->{
\TSetTick{ab}{0}{00}
\TSetTick{ab}{1}{01}
\TSetTick{ab}{2}{10}
\TSetTick{ab}{3}{11}
\TSort{(3,0.5)}{ab}{4}{l}
}

% Arcs de màj noirs de la sc
\only<5>{
\THit{a_1}{thick}{ab_0}{.west}{ab_2}
\THit{a_1}{thick}{ab_1}{.west}{ab_3}
\path[bounce,thick,bend left] \TBounce{ab_0}{thick}{ab_2}{.south} \TBounce{ab_1}{thick}{ab_3}{.south} ;
}

\only<6>{
\THit{a_0}{thick}{ab_2}{.west}{ab_0}
\THit{a_0}{thick}{ab_3}{.west}{ab_1}
\path[bounce,thick,bend right] \TBounce{ab_2}{thick}{ab_0}{.north} \TBounce{ab_3}{thick}{ab_1}{.north} ;
}

\only<7>{
\THit{b_0}{thick}{ab_3}{.west}{ab_2}
\THit{b_0}{thick}{ab_1}{.west}{ab_0}
\THit{b_1}{thick}{ab_0}{.west}{ab_1}
\THit{b_1}{thick}{ab_2}{.west}{ab_3}
\path[bounce,thick,bend right] \TBounce{ab_1}{thick}{ab_0}{.north} \TBounce{ab_3}{thick}{ab_2}{.north} ;
\path[bounce,thick,bend left] \TBounce{ab_0}{thick}{ab_1}{.south} \TBounce{ab_2}{thick}{ab_3}{.south} ;
}

% État d'exemple pour màj de la sc
\TState{8-9}{a_0,b_0}
\TState{10}{a_0,b_0,ab_0,ab_1,ab_2,ab_3}
\TState{11}{a_0,b_0,ab_0}
\only<9-11>{
\THit{a_1}{}{ab_0}{.west}{ab_2}
\THit{a_1}{}{ab_1}{.west}{ab_3}
\THit{b_0}{}{ab_3}{.west}{ab_2}
\THit{b_0}{}{ab_1}{.west}{ab_0}
\path[bounce,bend left] \TBounce{ab_0}{}{ab_2}{.south} \TBounce{ab_1}{}{ab_3}{.south} ;
\path[bounce,bend right] \TBounce{ab_1}{}{ab_0}{.north} \TBounce{ab_3}{}{ab_2}{.north} ;
}

% État d'exemple pour action de la sc
\TState{12}{a_0,b_0,z_2,ab_0}
\TState{13-14}{a_0,b_0,z_1,ab_0}

% Arc sortant de la sc
\only<12-14>{
\THit{ab_0}{thick}{z_2}{.west}{z_1}
\path[bounce,bend left,thick] \TBounce{z_2}{thick}{z_1}{.south} ;
}

% Arc sortant de la sc
\only<15->{
\THit{ab_0}{}{z_2}{.west}{z_1}
\path[bounce,bend left] \TBounce{z_2}{}{z_1}{.south} ;
}

}


%exple PH HM
%%% Exemple pour l'inférence %%%
\def \exphHM {
% Sortes
\TSort{(0,3)}{a}{2}{l}
\TSort{(0,0)}{b}{2}{l}
\TSort{(6,0)}{z}{3}{r}

% Sorte coopérative et arcs
\TSetTick{ab}{0}{00}
\TSetTick{ab}{1}{01}
\TSetTick{ab}{2}{10}
\TSetTick{ab}{3}{11}
\TSort{(3,0)}{ab}{4}{l}

% Actions de màj grisées
\THit{a_1}{ulhit}{ab_0}{.west}{ab_2}
\THit{a_1}{ulhit}{ab_1}{.west}{ab_3}
\path[bounce,bend left,pulhit] \TBounce{ab_0}{bulhit}{ab_2}{.south} \TBounce{ab_1}{bulhit}{ab_3}{.south};

\THit{a_0}{ulhit}{ab_2}{.west}{ab_0}
\THit{a_0}{ulhit}{ab_3}{.west}{ab_1}
\path[bounce,bend right,pulhit] \TBounce{ab_2}{bulhit}{ab_0}{.north} \TBounce{ab_3}{bulhit}{ab_1}{.north};

\THit{b_0}{ulhit}{ab_3}{.west}{ab_2}
\THit{b_0}{ulhit}{ab_1}{.west}{ab_0}
\THit{b_1}{ulhit}{ab_0}{.west}{ab_1}
\THit{b_1}{ulhit}{ab_2}{.west}{ab_3}
\path[bounce,bend right,pulhit] \TBounce{ab_1}{bulhit}{ab_0}{.north} \TBounce{ab_3}{bulhit}{ab_2}{.north};
\path[bounce,bend left,pulhit] \TBounce{ab_0}{bulhit}{ab_1}{.south} \TBounce{ab_2}{bulhit}{ab_3}{.south};

% Arcs sortant de la sc
\THit{ab_2}{ulhit}{z_1}{.north west}{z_2}
\THit{ab_2}{ulhit}{z_0}{.west}{z_1}
\path[bounce,bend left,pulhit] \TBounce{z_1}{bulhit}{z_2}{.south} \TBounce{z_0}{bulhit}{z_1}{.south};

\THit{ab_3}{ulhit}{z_2}{.west}{z_1}
\THit{ab_3}{ulhit}{z_0}{.west}{z_1}
\THit{ab_1}{ulhit}{z_2}{.west}{z_1}
\THit{ab_1}{ulhit}{z_0}{.west}{z_1}
\path[bounce,bend left,pulhit] \TBounce{z_2}{bulhit,bend right}{z_1}{.north};

\THit{ab_0}{ulhit}{z_2}{.west}{z_1}
\THit{ab_0}{ulhit}{z_1}{.south west}{z_0}
\path[bounce,bend right,pulhit] \TBounce{z_2}{bulhit}{z_1}{.north} \TBounce{z_1}{bulhit}{z_0}{.north};

}


% Exemples

% Exemple des définitions + points fixes
\def \exdef {
\TSort{(0,0)}{z}{3}{l}
\TSort{(3,3)}{b}{2}{t}
\TSort{(6,0)}{a}{2}{r}

\THit{b_0}{}{z_1}{.east}{z_2}
\THit{b_1}{}{z_0}{.east}{z_2}
\THit{a_0}{}{b_1}{.south}{b_0}
\THit{a_1}{out=60,in=0,selfhit}{a_1}{.east}{a_0}

\path[bounce,bend right]
\TBounce{z_1}{}{z_2}{.south}
\TBounce{z_0}{bend right=50}{z_2}{.south east}
;
\path[bounce,bend left]
\TBounce{a_1}{}{a_0}{.north}
\TBounce{b_1}{}{b_0}{.south}
;
}

% Idem réorganisé pour Points Fixes
\def \exdefb {
\path[use as bounding box] (0,-1) rectangle (4,4);

\TSort{(0,0)}{z}{3}{l}
\TSort{(2,4)}{b}{2}{t}
\TSort{(4,1)}{a}{2}{r}
}

% Frappes
\def \exdefbfrappes {
\THit{b_0}{}{z_1}{.east}{z_2}
\THit{b_1}{}{z_0}{.east}{z_2}
\THit{a_0}{}{b_1}{.south}{b_0}
\THit{a_1}{out=60,in=0,selfhit}{a_1}{.east}{a_0}

\path[bounce,bend right]
\TBounce{z_1}{}{z_2}{.south}
\TBounce{z_0}{bend right=50}{z_2}{.south east}
;
\path[bounce,bend left]
\TBounce{a_1}{}{a_0}{.north}
\TBounce{b_1}{}{b_0}{.south}
;
}

% Non-frappes
\def \exdefbsf {
\path[use as bounding box] (0,-1) rectangle (4,4);
\node[process,draw=red,thick] (a_1) at (a_1.center) {};

\path<2,4-> (z_0) edge (b_0) edge (a_0) (b_0) edge (a_0);
\path<2-3> (z_2) edge (b_0) edge (a_0);
\path<3>[very thick] (z_0) edge (b_0) edge (a_0) (b_0) edge (a_0);
\path<4->[very thick] (z_2) edge (b_0) edge (a_0) (b_0) edge (a_0);
\TState{3}{z_0,b_0,a_0}
\TState{4-}{z_2,b_0,a_0}

\path (z_2) edge (b_1);
\path (z_1) edge (b_1);
\path (z_1) edge (a_0);
}



% Figure de présentation de l'analyse d'atteignabilité
\def \figsa {
\begin{tikzpicture}
\path[use as bounding box] (-5,-2.6) rectangle (5,2.8);
\definecolor{r2}{RGB}{238,10,38}

\path<2->[shading=1, inner color=r2, outer color=white] (3.5,-2.8) -- (4.4,3.2) -- (0,3) -- (-4.5,1.4) -- (-2.5,-2.5) -- (0,-3.6) -- (2.8,-2.8);
%\path<2->[shading, inner color=r2, outer color=white, border color=white] (2.8,-2.8) -- (4.5,4.5) -- (0,3.9) -- (-4.5,1.8) -- (-5,-3) -- (0,-3.2) -- (2.8,-2.8);
\draw<2->[thick,fill=white] (2.5,-2.1) -- (3,2.5) -- (-2.7,1.3) -- (-2,-2) -- (2.5,-2.1);
\draw<6->[thick,fill=lightyellow] (2.5,-2.1) -- (3,2.5) -- (-2.7,1.3) -- (-2,-2) -- (2.5,-2.1);

\node<2->[text width=3.5cm, color=red] (s1) at (-5,2) {Over-Approximation};
\path<2->[->,very thick,color=red] (s1.south) edge (-3.5,1.2);
%\node<2->[text width=3cm,color=black] (i1) at (3.7,.2) {$\Rightarrow$};
\node<2->[text width=3cm,color=black] (q) at (4.5,.2) {$\neg Q$};

\draw<4->[thick, fill=green] (.5,-.8) -- (1,0) -- (.3,1) -- (-1,.5) -- (-.5,-.5) -- (.5,-.8);
\node<4->[text width=3.5cm,color=darkgreen] (s2) at (5.2,-1.5) {Under-Approximation};
\node<4->[text width=3cm,color=black] (p) at (1.8,.2) {$P$};
%\node<4->[text width=3cm,color=black] (i1) at (2.25,.2) {$\Rightarrow$};

% reaching set
\node[text width=3cm,color=darkcyan] (s) at (1.8,1.7) {Exact solution};
\node<1->[text width=3cm,color=darkcyan] (s0) at (0,0) {};
\draw[color=darkcyan, thick] (0,0) ellipse (2 and 1.5);
%\path<1>[draw=white] (2.8,-2.8) -- (4.5,4.5) -- (0,3.9) -- (-4.5,1.8) -- (-5,-3) -- (-2.5,-3.5) -- (0,-3.2) -- (2.8,-2.8);
\node[text width=3cm,color=black] (r) at (2.8,.2) {$R$};

\path<4->[->,very thick,color=darkgreen] (s2) edge (.6,-.4);

\tikzstyle{point}=[circle,draw=blue,fill=blue,minimum size=5pt,inner sep=0pt]

%\only<5->{
\only<3->{
\node[point] at (-2.4,-2) {};
\node[point] at (-2,2) {};
}
\only<5->{
\node[point] at (0,0) {};
}
\only<7->{
\node[point] at (-.5,-1.1) {};
\node[point] at (2.5,1) {};
}
%}

\end{tikzpicture}
}



% Exemple atteignabilité
\def \exatt {
\path[use as bounding box] (-1,-3) rectangle (7,2);
\TSort{(0,0)}{a}{2}{l}
\TSort{(3,0)}{b}{3}{l}
\TSort{(6,0)}{d}{3}{r}
\TSort{(2,-2)}{c}{2}{b}

\THit{a_0}{}{c_0}{.north}{c_1}
\THit{a_1}{}{b_1}{.west}{b_0}
\THit{c_1}{bend left=20pt}{b_0}{.west}{b_1}
\THit{b_1.south west}{->}{a_0}{.east}{a_1}
\THit{b_0}{}{d_0}{.west}{d_1}
\THit{b_1}{}{d_1}{.west}{d_2}
\THit{d_1}{}{b_0}{.north east}{b_2}
\THit{c_1}{bend right=80pt,distance=80pt}{d_1}{.east}{d_0}
\THit{b_2}{distance=120pt,out=30,in=40}{d_0}{.east}{d_2}

\path[bounce,bend left]
\TBounce{d_0}{}{d_1}{.south}
\TBounce{d_1}{}{d_2}{.south}
\TBounce{c_0}{}{c_1}{.west}
\TBounce{b_0}{}{b_1}{.south}
\TBounce{d_1}{}{d_0}{.north}
;
\path[bounce,bend right]
\TBounce{a_0}{}{a_1}{.south}
\TBounce{b_0}{}{b_2}{.south}
\TBounce{b_1}{}{b_0}{.north}
\TBounce{d_0}{bend right=50pt,distance=40pt}{d_2}{.south}
;
}


%Exemple atteignabilité avec les rates
\def \exattnew {

 \TSort{(0,0)}{a}{2}{l}
      \TSort{(2,0)}{b}{2}{l}
      \TSort{(2,3)}{bs}{2}{l}
      \TSort{(3.5,-3)}{c}{2}{l}
      \TSort{(4,0)}{d}{3}{l}
      \TSort{(6,0)}{e}{2}{l}

      %\TSetTick{ab}{0}{00}
      %\TSetTick{ab}{1}{01}
      %\TSetTick{ab}{2}{10}
      %\TSetTick{ab}{3}{11}
      %\TSort{(4,1)}{ab}{4}{r}

      %\THit{a_0}{prio}{ab_3}{.west}{ab_1}
      %\THit{a_0}{prio}{ab_2}{.south west}{ab_0}
      \THit{a_1}{}{b_0}{.west}{b_1}
      \THit{a_1}{}{bs_0}{.west}{bs_1}
      \THit{a_0}{}{c_0}{.west}{c_1}

      %\THit{a_1}{prio}{ab_0}{.south west}{ab_2}

      %\THit{b_0}{prio}{ab_3}{.north west}{ab_2}
      %\THit{b_0}{prio}{ab_1}{.north west}{ab_0}
      \THit{b_1}{}{d_1}{.west}{d_2}
      \THit{b_0}{}{d_0}{.west}{d_1}
      \THit{bs_1}{}{d_1}{.west}{d_2}
      \THit{c_1}{}{b_1}{.south east}{b_0}
      \THit{e_1}{}{d_1}{.east}{d_0}
      \THit{e_0}{selfhit}{e_0}{.south}{e_1}
      
      %\THit{ab_3}{}{c_0}{.west}{c_1}
      
      
      \path[bounce, bend left]
        \TBounce{b_0}{}{b_1}{.south west}
      ;
      \path[bounce, bend left]
        \TBounce{bs_0}{}{bs_1}{.south west}
      ;
      \path[bounce, bend left]
        \TBounce{c_0}{}{c_1}{.south west}
      ;
      \path[bounce, bend left]
        \TBounce{d_1}{}{d_2}{.south west}
      ;
      \path[bounce, bend left]
        \TBounce{d_0}{}{d_1}{.south west}
      ;
      \path[bounce, bend left]
        \TBounce{b_1}{}{b_0}{.north east}
      ;
      \path[bounce, bend left]
        \TBounce{d_1}{}{d_0}{.north east}
      ;
      \path[bounce, bend right]
        \TBounce{e_0}{}{e_1}{.south east}
      ;
      
     % \TAction{a_1}{a_1.west}{a_0.north west}{selfhit}{right}
     % \TAction{b_1}{b_1.west}{b_0.north west}{selfhit}{right}
     % \TAction{a_0.south west}{b_0.west}{b_1.south west}{bend left=90}{left}
     % \TAction{b_0}{a_0.west}{a_1.south west}{bend right=50}{left}

     % on rajoute les labels sur les arcs
     
     \node[labelprio1] at (3,3.15) {$0.5$}; % bs 1-> d 1 2
     \node[labelprio1] at (2.80,1.2) {$0.7$};    % b 1 -> d 1 2
     \node[labelprio1] at (0.55,2) {$0.8$}; % a 1-> bs 0 1
     \node[labelprio1] at (0.80,1) {$0.6$};    % a 1 -> b 0 1
     \node[labelprio2] at (2.90,-0.3) {$10$};    % c 1 -> b 1 0
     \node[labelprio2] at (4.80,1.2) {$7$};    % e 1 -> d 1 0


}


% Exemple atteignabilité
\def \exattbis {
\path[use as bounding box] (-1,-3) rectangle (7,2);
\TSort{(0,-1)}{a}{2}{l}
\TSort{(9,1)}{f}{2}{l}
\TSort{(3,0)}{b}{3}{l}
\TSort{(6,0)}{d}{3}{r}
\TSort{(2,-2)}{c}{2}{b}

\THit{a_0}{}{c_0}{.north}{c_1}
\THit{a_1}{}{b_1}{.west}{b_0}
\THit{c_1}{bend left=20pt}{b_0}{.west}{b_1}
\THit{b_1.south west}{->}{a_0}{.east}{a_1}
\THit{b_0}{}{d_0}{.west}{d_1}
\THit{b_1}{}{d_1}{.west}{d_2}
\THit{d_1}{}{b_0}{.north east}{b_2}
\THit{c_1}{bend right=80pt,distance=80pt}{d_1}{.east}{d_0}
%\THit{b_2}{distance=120pt,out=30,in=40}{d_0}{.east}{d_2}

\path[bounce,bend left]
\TBounce{d_0}{}{d_1}{.south}
\TBounce{d_1}{}{d_2}{.south}
\TBounce{c_0}{}{c_1}{.west}
\TBounce{b_0}{}{b_1}{.south}
\TBounce{d_1}{}{d_0}{.north}
;
\path[bounce,bend right]
\TBounce{a_0}{}{a_1}{.south}
\TBounce{b_0}{}{b_2}{.south}
\TBounce{b_1}{}{b_0}{.north}
%\TBounce{d_0}{bend right=50pt,distance=40pt}{d_2}{.south}
;
}




% Structure abstraite / Sous-approximation / Ok
\def \sauyes {%
\begin{tikzpicture}[aS,node distance=1.1cm,shorthandon]
\path[use as bounding box] (-0.5,-2.1) rectangle (10.25,2.2);

\node[Aobj] (d02) {$\PHobjectif{d_0}{d_2}$};
\node[Aproc,above of=d02] (d2) {$d_2$};

\node[Asol,right of=d02] (d02s2) {};
\node[Aproc,above right of=d02s2] (b0) {$b_0$};
\node[Aobj,right of=b0] (b10) {$\PHobjectif{b_1}{b_0}$};
\node[Asol,right of=b10] (b10s) {};
\node[Aproc,right of=b10s] (a1) {$a_1$};
\node[Aobj,right of=a1] (a11) {$\PHobjectif{a_1}{a_1}$};
\node[Asol,right of=a11] (a11s) {};

\node[Aobj,above of=b10,yshift=-0.5cm] (b00)
{$\PHobjectif{b_0}{b_0}$};
\node[Asol,right of=b00] (b00s) {};

\node[Aproc, below of=b0] (b1) {$b_1$};
\node[Aobj,right of=b1] (b11) {$\PHobjectif{b_1}{b_1}$};
\node[Asol,right of=b11] (b11s) {};
\node[Aobj,below of=b11] (b01) {$\PHobjectif{b_0}{b_1}$};
\node[Asol,right of=b01] (b01s) {};
\node[Aproc,right of=b01s] (c1) {$c_1$};
\node[Aobj,right of=c1] (c11) {$\PHobjectif{c_1}{c_1}$};
\node[Asol,right of=c11] (c11s) {};

\path
(d02) edge (d02s2) (d02s2) edge (b1) edge (b0)
(a11) edge (a11s)
(b10) edge (b10s) (b10s) edge (a1)
(b11) edge (b11s)
(b0) edge (b10) (b1) edge (b11)
(a1) edge (a11)
(d2) edge (d02)
;
\path
(b0) edge (b00.west) (b00) edge (b00s)
(b1) edge (b01)
(b01) edge (b01s) (b01s) edge (c1)
(c1) edge (c11) (c11) edge (c11s)
;
%\node<\tu>[right of=a11s] {\textbf{\Large\color{darkgreen}Yes}};
\end{tikzpicture}%
}

% Structure abstraite / Sous-approximation / Inconclusif
\def \sauinconc {%
\begin{tikzpicture}[aS,node distance=1.1cm,shorthandon]
\path[use as bounding box] (-0.5,-2.1) rectangle (10.25,2.2);

\node[Aobj] (d02) {$\PHobjectif{d_0}{d_2}$};
\node[Aproc,above of=d02] (d2) {$d_2$};

\node[Asol,right of=d02] (d02s2) {};
\node[Aproc,above right of=d02s2] (b0) {$b_0$};
\node[Aobj,right of=b0] (b10) {$\PHobjectif{b_1}{b_0}$};
\node[Asol,right of=b10] (b10s) {};
\node[Aproc,right of=b10s] (a1) {$a_1$};
\node[Aobj,right of=a1] (a01) {$\PHobjectif{a_0}{a_1}$};
\node[Asol,right of=a01] (a01s) {};

\node[Aproc, below of=b0] (b1) {$b_1$};
\node[Aobj,right of=b1] (b11) {$\PHobjectif{b_1}{b_1}$};
\node[Asol,right of=b11] (b11s) {};
\node[Aobj,below of=b11] (b01) {$\PHobjectif{b_0}{b_1}$};
\node[Asol,right of=b01] (b01s) {};
\node[Aproc,right of=b01s] (c1) {$c_1$};
\node[Aobj,right of=c1] (c01) {$\PHobjectif{c_0}{c_1}$};
\node[Asol,right of=c01] (c01s) {};
\node[Aproc,right of=c01s] (a0) {$a_0$};
\node[Aobj,right of=a0] (a00) {$\PHobjectif{a_0}{a_0}$};
\node[Asol,right of=a00] (a00s) {};

\node[Aobj,above of=b10] (b00) {$\obj{b_0}{b_0}$};
\node[Asol,right of=b00] (b00s) {};
\node[Aobj,above of=a01] (a11) {$\obj{a_1}{a_1}$};
\node[Asol,right of=a11] (a11s) {};
\node[Aobj,above of=c01] (c11) {$\obj{c_1}{c_1}$};
\node[Asol,right of=c11] (c11s) {};
\node[Aobj,above of=a00] (a10) {$\PHobjectif{a_1}{a_0}$};
\node at (a10.east) {\Large\color{red}\textbf{$\bot$}};

\path
  (b10) edge[loop,min distance=5mm] (b10)
 ;
\path
(d02) edge (d02s2) (d02s2) edge (b1) edge (b0)
(a01) edge (a01s) (a01s.south) edge (b1.north east)
(b10) edge (b10s) (b10s) edge (a1)
(b11) edge (b11s)
(a1) edge (a01)
(b0) edge (b10) (b1) edge (b11)
(d2) edge (d02)
;
\path
(b00) edge (b00s)
(b0) edge (b00)
 (b1) edge (b01)
 (b01) edge (b01s) (b01s) edge (c1)
 (c1) edge (c01)
 (c01) edge (c01s) (c01s) edge (a0)
 (a0) edge (a00) (a00) edge (a00s)
;
\path
 (c1) edge (c11) (c11) edge (c11s)
(a0) edge (a10)
(a1) edge (a11)
(a11) edge (a11s)
;

%\node[right of=a01s] {\textbf{\Large\color{darkyellow}Inconc}};

\end{tikzpicture}%
}

% Structure abstraite / Sur-approximation / Non
\def \saono {%
\begin{tikzpicture}[aS,node distance=1.1cm,shorthandon]
\path[use as bounding box] (-0.5,-2.1) rectangle (10.25,1.15);

\node[Aobj] (d12) {$\PHobjectif{d_1}{d_2}$};
\node[Asol,above right of=d12] (d12s1) {};
\node[Aproc, right of=d12s1] (b2) {$b_2$};
\node[Aobj,right of=b2] (b02) {$\PHobjectif{b_0}{b_2}$};
\node[Asol,right of=b02] (b02s) {};
\node[Aproc,right of=b02s] (d1) {$d_1$};
\node[Aobj,right of=d1] (d11) {$\PHobjectif{d_1}{d_1}$};
\node[Asol,right of=d11] (d11s) {};

\node[Asol,below right of=d12] (d12s2) {};
\node[Aproc, right of=d12s2] (b1) {$b_1$};
\node[Aobj,right of=b1] (b01) {$\PHobjectif{b_0}{b_1}$};
\node[Asol,right of=b01] (b01s) {};
\node[Aproc,right of=b01s] (c1) {$c_1$};
\node[Aobj,right of=c1] (c01) {$\PHobjectif{c_0}{c_1}$};
\node[Asol,right of=c01] (c01s) {};
\node[Aproc,right of=c01s] (a0) {$a_0$};
\node[Aobj,right of=a0] (a10) {$\PHobjectif{a_1}{a_0}$};
\node at (a10.east) {\Large\color{red}\textbf{$\bot$}};

\path
(d12) edge (d12s1) edge (d12s2) (d12s1) edge (b2) edge (c1) (d12s2) edge (b1)
(b01) edge (b01s) (b01s) edge (c1)
(b02) edge (b02s) (b02s) edge (d1)
(c01) edge (c01s) (c01s) edge (a0)
(d11) edge (d11s)
(a0) edge (a10)
(b1) edge (b01)
(b2) edge (b02)
(c1) edge (c01)
(d1) edge (d11)
;
%\only<\value{anim1}>{ \node[above right of=c01s] {\textbf{\Large\color{red}No}};}
\end{tikzpicture}%
}

% Structure abstraite / Sur-approximation / Inconclusif
\def \saoinconc {%
\begin{tikzpicture}[aS,node distance=1.1cm,shorthandon]
\path[use as bounding box] (-0.5,-2.1) rectangle (10.25,1.15);

\node[Aobj] (d02) {$\PHobjectif{d_0}{d_2}$};
\node[Asol,above right of=d02] (d02s1) {};

\node[Aproc, right of=d02s1] (b2) {$b_2$};
\node[Aobj,right of=b2] (b12) {$\PHobjectif{b_1}{b_2}$};
\node[Asol,right of=b12] (b12s) {};
\node[Aproc,right of=b12s] (d1) {$d_1$};
\node[Aobj,right of=d1] (d01) {$\PHobjectif{d_0}{d_1}$};
\node[Asol,right of=d01] (d01s) {};

\node[Asol,below right of=d02] (d02s2) {};
%<-3>
\node<-\tof>[Aproc, right of=d02s2] (b0) {$b_0$};
\node<\tokp>[orange, thick, Aproc, right of=d02s2] (b0) {$b_0$};
\node[Aobj,right of=b0] (b10) {$\PHobjectif{b_1}{b_0}$};
\node[Asol,right of=b10] (b10s) {};
%<-3>
\node<-\tof>[Aproc,right of=b10s] (a1) {$a_1$};
\node<\tokp>[orange, thick, Aproc,right of=b10s] (a1) {$a_1$};
\node[Aobj,right of=a1] (a11) {$\PHobjectif{a_1}{a_1}$};
\node[Asol,right of=a11] (a11s) {};

\node[Aproc, below of=b0] (b1) {$b_1$};
\node[Aobj,right of=b1] (b11) {$\PHobjectif{b_1}{b_1}$};
\node[Asol,right of=b11] (b11s) {};

\node<\tokp>[orange, font=\bfseries,below of=a11s] (kp) {Key processes};
\path<\tokp>[orange, thick]
        (kp) edge (a1)
        (kp) edge (b0)
;
\path
(d02) edge (d02s1) edge (d02s2) (d02s1) edge (b2) (d02s2) edge (b1) edge (b0)
(a11) edge (a11s)
(b10) edge (b10s) (b10s) edge (a1)
(b11) edge (b11s)
(b12) edge (b12s) (b12s) edge (d1) edge (a1)
(d01) edge (d01s) (d01s.south) edge (b0)
(a1) edge (a11)
(b0) edge (b10) (b1) edge (b11) (b2) edge (b12)
(d1) edge (d01)
;
%\node[below right of=d01s] {\textbf{\Large\color{yellow}Inconc}};
\end{tikzpicture}%
}


\def \sasaquant {

    \begin{tikzpicture}[aS,node distance=1.1cm,shorthandon]
    
    % \path[use as bounding box] (-0.5,-3.1) rectangle (10.25,1.15);
     \path[use as bounding box] (3,0) rectangle (10.25,-2.15);

      \node[Aproc] (d2) {$d_2$};
      \node[Aobj,below of=d2] (d12) {$\PHobjectif{d_1}{d_2}{\color{blue}(0.14)}$};
      \node[Asol,right of=d12] (d12s) {};
      

      \node[Aproc,right of=d12s] (bbs) {$b_1,bs_1,\rlab{e_1}$};
      \node[Asol,right of=bbs] (bbss) {};

      \node[Aproc,right  of=bbss] (b1) {$b_1$};
      \node[Aobj,right of=b1] (b01) {$\PHobjectif{b_0}{b_1}{\color{blue}(0.05)}$};
      \node[Asol,right of=b01] (b01s) {};
      %\node[Aobj,below left of=a1] (a01) {$\PHobj{a_0}{a_1}$};
      %\node[Asol,below of=a01] (a01s) {};
      \node[Aproc,right of=b01s] (a1) {$\rlab{c_1},a_1$};
      \node[Aobj,right of=a1] (a11) {$\PHobjectif{a_1}{a_1}{\color{blue}(1)}$};
      \node[Asol,right of=a11] (a11s) {};
      \node[RAobj,above right of=a1] (c11) {$\PHobjectif{c_1}{c_1}{\color{blue}(1)}$};
      \node[RAsol,right of=c11] (c11s) {{\Large\color{red}\textbf{$\otimes$}}};
      %\node[Aobj,below left of=b0] (b10) {$\PHobj{b_1}{b_0}$};
      %\node[Asol,below of=b10] (b10s) {};

      \node[Aproc,below right  of=bbss] (bs1) {$bs_1$};
      \node[Aobj,right of=bs1] (bs01) {$\PHobjectif{bs_0}{bs_1}{\color{blue}(1)}$};
      \node[Asol,right of=bs01] (bs01s) {};
      %\node[Aobj,below right of=b1] (b01) {$\PHobj{b_0}{b_1}$};
      %\node[Asol,below of=b01] (b01s) {};
      \node[Aproc,right of=bs01s] (as1) {$a_1$};
      \node[Aobj,right of=as1] (as11) {$\PHobjectif{a_1}{a_1}{\color{blue}(1)}$};
      \node[Asol,right of=as11] (as11s) {};
      %\node[Aobj,below right of=a0] (a10) {$\PHobj{a_1}{a_0}$};
      %\node[Asol,below of=a10] (a10s) {};

      \node[RAproc,above right of=bbss] (e1) {$e_1$};
      \node[RAobj,right of=e1] (e01) {$\PHobjectif{e_0}{e_1}{\color{blue}(1)}$};
      \node[RAsol,right of=e01] (e01s) {{\Large\color{red}\textbf{$\otimes$}}};

      \path
      (d2) edge (d12)
      (d12) edge (d12s)
      (d12s) edge (bbs)
      (bbs) edge (bbss)
      (bbss) edge (bs1) edge (b1) edge[red] (e1)

      (b1) edge (b01) 
      (b01) edge (b01s)
      (b01s) edge (a1)
      (a1) edge (a11)
      (a11) edge (a11s)
      (a1) edge[red] (c11)
      (c11) edge[red] (c11s)
      %(a0) edge (a10) edge (a00)
      %(a10) edge (a10s)
      %(a00) edge (a00s)

      (bs1) edge (bs01) 
      (bs01) edge (bs01s)
      (bs01s) edge (as1)
      (as1) edge (as11)
      (as11) edge (as11s) 
      %(b01) edge (b01s)
      %(b01s) edge (a0)
      %(b11) edge (b11s)

      (e1) edge[red] (e01) 
      (e01) edge[red] (e01s)
      ;
      \end{tikzpicture}


}

%%% Exemple pour la définition des an %%%
\def \exandef {
\path[use as bounding box] (-0.5,-0.5) rectangle (6.5,4.5);

\TSort{(0,1)}{a}{3}{l}
\TSort{(3,1)}{b}{2}{l}
\TSort{(6,1)}{c}{3}{l}

\path[local transitions]
  (a_0) edge node[auto] {$b_0$} (a_1)
  (a_1) edge (a_0)
  (a_0) edge[bend right=60] node[right] {$b_0,c_0$} (a_2)
  (c_0) edge node[auto] {$a_1$} (c_1)
  (c_1) edge node[auto] {$b_0$} (c_2)
  (c_1) edge node[auto] {$b_1$} (c_0)
  (b_0) edge (b_1)
  (b_1) edge node[auto] {$a_0$} (b_0)
;
}

%%% Exemple pour la définition des an %%%
\def \exanaspdef {
\path[use as bounding box] (-0.5,-0.5) rectangle (6.5,4.5);

\TSort{(3,1.5)}{a}{3}{l}

\path[local transitions]
  (a_0) edge node[auto] {$b_0$} (a_1)
  (a_1) edge (a_0)
  (a_0) edge[bend right=60] node[right] {$b_0,c_0$} (a_2)
;
}

%%% Exemple pour la définition des san %%%
\def \exsandef {
\path[use as bounding box] (-0.5,-0.5) rectangle (6.5,4.5);

\TSort{(0,1)}{a}{3}{l}
\TSort{(3,1)}{b}{2}{l}
\TSort{(6,1)}{c}{3}{l}

\path[local transitions]
  (a_0) edge node[auto] {$b_0,\mathbf{\color{Maroon} 2}$} (a_1)
  (a_1) edge node[auto] {$\mathbf{\color{Maroon} 1}$} (a_0)
  (a_0) edge[bend right=60] node[right] {$b_0,c_0,\mathbf{\color{Maroon} 2}$} (a_2)
  (c_0) edge node[auto] {$a_1,\mathbf{\color{Maroon} 3}$} (c_1)
  (c_1) edge node[auto] {$b_0,\mathbf{\color{Maroon} 2}$} (c_2)
  (c_1) edge node[auto] {$b_1,\mathbf{\color{Maroon} 1}$} (c_0)
  (b_0) edge node[auto] {$\mathbf{\color{Maroon} 3}$}(b_1)
  (b_1) edge node[auto] {$a_0,\mathbf{\color{Maroon} 1}$} (b_0)
;
}

%%% Exemple pour l'interpretation abstraite %%%
\def \exanaidef {
\path[use as bounding box] (-0.5,-0.5) rectangle (6.5,4.5);

\TSort{(3,1.5)}{d}{3}{l}

\path[local transitions]
  (d_1) edge[bend right] node[right] {$c_2$} (d_2)
  (d_1) edge node[auto] {$b_1$} (d_2)
  (d_1) edge node[auto] {$e_1$} (d_0)
;
}

%le GLC associé
\def \exlcgaidef {

    \begin{tikzpicture}[aS,node distance=1.1cm,shorthandon]
    
    \path[use as bounding box] (0,-1) rectangle (2,1);
    %les noeuds
    \node[Aproc] (d2) {$d_2$};
    \node[Aobj,below of=d2] (d12) {$\obj{d_1}{d_2}$};
    \node[Asol,below left of=d12] (d12s1) {};
    \node[Asol,below right of=d12] (d12s2) {};

    \node[Aproc, below right of=d12s2] (c2) {$c_2$};
    \node[Aproc, below left of=d12s1] (b1) {$b_1$};
    %\node[Aproc, below right of=a02s2] (b0) {$b_0$};
    
    

    \path
      (d2) edge (d12)
      (d12) edge (d12s1)
      (d12) edge (d12s2)
      (d12s1) edge (b1)
      (d12s2) edge (c2)
;
\end{tikzpicture}
} 


%%% Exemple pour l'interpretation abstraite quantifié%%%
\def \exsanaidef {
\path[use as bounding box] (1.5,-0.5) rectangle (3.5,3.5);

\TSort{(3,1.5)}{d}{3}{l}

\path[local transitions]
  (d_1) edge[bend right] node[right] {$c_2,\mathbf{\color{Maroon} 2}$} (d_2)
  (d_1) edge node[auto] {$b_1,\mathbf{\color{Maroon} 3}$} (d_2)
  (d_1) edge[red] node[auto] {${\color{red}e_1},\mathbf{\color{Maroon} 2}$} (d_0)
;
}

%le QGLC associé
\def \exqlcgaidef {

    \begin{tikzpicture}[aS,node distance=1.1cm,shorthandon]
    
    \path[use as bounding box] (0,-3) rectangle (2,1);
    %les noeuds
    \node[Aproc] (d2) {$d_2$};
    \node[Aobj,below of=d2] (d12) {$\obj{d_1}{d_2}$};
    \node[Asol,below left of=d12] (d12s1) {};
    \node[Asol,below right of=d12] (d12s2) {};

    \node[Aproc, below right of=d12s2] (c2) {$c_2$};
    \node[Aproc, below left of=d12s1] (b1) {$b_1$};
    %\node[Aproc, below right of=a02s2] (b0) {$b_0$};
    \node[labelproba, right of=d2,node distance=1.8cm] (pd2) {${\color{blue}\{\Prob{d_2},\Time{d_2}\}}$};
    \node[labelproba, right of=d12,node distance=2.8cm] (pd12) {${\color{blue}\{\Prob{\obj{d_1}{d_2}},\Time{\obj{d_1}{d_2}}\}}$};
    \node[labelproba, left of=b1,node distance=1.5cm] (pb1) {${\color{blue}\{\Prob{b_1},\Time{b_1}\}}$};
    \node[labelproba, right of=c2,node distance=1.5cm] (pc2) {${\color{blue}\{\Prob{c_2},\Time{c_2}\}}$};
    
    
    \path
      (d2) edge (d12)
      (d12) edge (d12s1)
      (d12) edge (d12s2)
      (d12s1) edge (b1)
      (d12s2) edge (c2)
;
\end{tikzpicture}
} 

%le QGLC généralisé
\def \exqlcggaidef {

    \begin{tikzpicture}[aS,node distance=1.1cm,shorthandon]
    
    \path[use as bounding box] (0,-3) rectangle (2,1);
    %les noeuds
    \node[Aproc] (d2) {$\n_k$};
    \node[Aobj,below of=d2] (d12) {$obj_k$};
    \node[Asol,below left of=d12] (d12s1) {};
    \node[Asol,below right of=d12] (d12s2) {};

    \node[Aproc, below right of=d12s2] (c2) {$\n_{k-1,2}$};
    \node[Aproc, below left of=d12s1] (b1) {$\n_{k-1,1}$};
    %\node[Aproc, below right of=a02s2] (b0) {$b_0$};
    \node[labelproba, right of=d2,node distance=1.8cm] (pd2) {${\color{blue}\{\Prob{\n_k},\Time{\n_k}\}}$};
    \node[labelproba, right of=d12,node distance=1.8cm] (pd12) {${\color{blue}\{\Prob{obj_k},\Time{obj_k}\}}$};
    \node[labelproba, left of=b1,node distance=2.4cm] (pb1) {${\color{blue}\{\Prob{\n_{k-1,1}},\Time{\n_{k-1,1}}\}}$};
    \node[labelproba, right of=c2,node distance=2.3cm] (pc2) {${\color{blue}\{\Prob{\n_{k-1,2}},\Time{\n_{k-1,2}}\}}$};
    
    
    \path
      (d2) edge (d12)
      (d12) edge (d12s1)
      (d12) edge (d12s2)
      (d12s1) edge (b1)
      (d12s2) edge (c2)
;
\end{tikzpicture}
} 



%exemple d'un an pour annoncer les san et le glc quantifié
\def \exanppdef {
%\path[use as bounding box] (-0.5,-0.5) rectangle (10.5,17.5);

\TSort{(0,1)}{a}{2}{l}
\TSort{(2.5,1)}{b}{2}{l}
\TSort{(2.5,4)}{c}{3}{l}
\TSort{(0,4)}{d}{3}{l}
\TSort{(2.5,8)}{e}{2}{l}
\TSort{(0,8)}{f}{2}{l}
%\TState{a_0,b_0,c_1,d_1,e_1,f_0}

\path[local transitions]
  (a_0) edge (a_1)
  (a_1) edge node[auto] {$c_2$}(a_0)
  (c_0) edge node[auto] {$f_0$} (c_1)
  (c_1) edge node[auto] {$f_0$} (c_2)
  (c_1) edge node[auto] {$f_1$} (c_0)
  (b_0) edge node[auto] {$a_1$}(b_1)
  (b_1) edge node[auto] {$a_0$} (b_0)
  (d_1) edge[bend right] node[right] {$c_2$} (d_2)
  (d_1) edge node[auto] {$b_1$} (d_2)
  (d_1) edge node[auto] {$e_1$} (d_0)
  (e_0) edge node[auto] {$d_0$} (e_1)
  (e_1) edge node[auto] {$d_2$} (e_0)
  (f_0) edge (f_1)
  (f_1) edge node[auto] {$b_1$} (f_0)
;
}

\def \exglcandef {

\begin{tikzpicture}[aS,node distance=1.1cm,shorthandon]
    
    %\path[use as bounding box] (0,-1) rectangle (2,1);
    %les noeuds
    \node[Aproc] (a2) {$d_2$};
    \node[Aobj,below of=a2] (a02) {$\obj{d_1}{d_2}$};
    \node[Asol,below left of=a02] (a02s1) {};
    \node[Asol,below right of=a02] (a02s2) {};

    \node[Aproc, below right of=a02s2] (c2) {$c_2$};
    \node[Aproc, below left of=a02s1] (b1) {$b_1$};
    
    \node[Aobj,below of=b1] (b01) {$\obj{b_0}{b_1}$};
    \node[Asol,below of=b01] (b01s) {};
    \node[Aproc,below of=b01s] (a1) {$a_1$};
    \node[Aobj,below of=a1] (a01) {$\obj{a_0}{a_1}$};
    \node[Asol,below of=a01] (a01s) {};
    
    
    \node[Aobj,below of=c2] (c12) {$\obj{c_1}{c_2}$};
    \node[Asol,below of=c12] (c12s) {};
    \node[Aproc,below of=c12s] (f0) {$f_0$};
    \node[Aobj,below of=f0] (f00) {$\obj{f_0}{f_0}$};
    \node[Asol,below of=f00] (f00s) {};
    

    \path
      (a2) edge (a02)
      (a02) edge (a02s1)
      (a02) edge (a02s2)
      (a02s1) edge (b1)
      (a02s2) edge (c2)
      (c2) edge (c12)
      (c12) edge (c12s)
      (c12s) edge (f0)
      (f0) edge (f00)
      (f00) edge (f00s)
      (b1) edge (b01)
      (b01) edge (b01s)
      (b01s) edge (a1)
      (a1) edge (a01)
      (a01) edge (a01s)
;
\end{tikzpicture}
}




%exemple d'un san pour illustrer le glc quantifié

\def \exsanppdef {
%\path[use as bounding box] (-0.5,-0.5) rectangle (10.5,17.5);

\TSort{(0,1)}{a}{2}{l}
\TSort{(2.5,1)}{b}{2}{l}
\TSort{(2.5,4)}{c}{3}{l}
\TSort{(0,4)}{d}{3}{l}
\TSort{(2.5,8)}{e}{2}{l}
\TSort{(0,8)}{f}{2}{l}


\path[local transitions]
  (a_0) edge node[auto] {$\mathbf{\color{Maroon} 2}$} (a_1)
  (a_1) edge node[auto] {$c_2,\mathbf{\color{Maroon} 1}$} (a_0)
  (c_0) edge node[auto] {$f_0,\mathbf{\color{Maroon} 2}$} (c_1)
  (c_1) edge node[auto] {$f_0,\mathbf{\color{Maroon} 2}$} (c_2)
  (c_1) edge node[auto] {$f_1,\mathbf{\color{Maroon} 1}$} (c_0)
  (b_0) edge node[auto] {$a_1,\mathbf{\color{Maroon} 2}$}(b_1)
  (b_1) edge node[auto] {$a_0,\mathbf{\color{Maroon} 1}$} (b_0)
  (d_1) edge[bend right] node[right] {$c_2,\mathbf{\color{Maroon} 2}$} (d_2)
  (d_1) edge node[auto] {$b_1,\mathbf{\color{Maroon} 3}$} (d_2)
  (d_1) edge node[auto] {$e_1,\mathbf{\color{Maroon} 2}$} (d_0)
  (e_0) edge node[auto] {$d_0,\mathbf{\color{Maroon} 2}$} (e_1)
  (e_1) edge node[auto] {$d_2,\mathbf{\color{Maroon} 1}$} (e_0)
  (f_0) edge node[auto] {\mathbf{\color{Maroon} 2}}(f_1)
  (f_1) edge node[auto] {$b_1,\mathbf{\color{Maroon} 1}$} (f_0)
;
}

%exemple de GLC quantifié 
\def \exglcsandef {

\begin{tikzpicture}[aS,node distance=1.1cm,shorthandon]
    
    %\path[use as bounding box] (0,-1) rectangle (2,1);
    %les noeuds
    \node[Aproc] (a2) {$d_2$};
    \node[Aobj,below of=a2] (a02) {$\obj{d_1}{d_2}$};
    \node[Asol,below left of=a02] (a02s1) {};
    \node[Asol,below right of=a02] (a02s2) {};

    \node[Aproc, below right of=a02s2] (c2) {$c_2$};
    \node[Aproc, below left of=a02s1] (b1) {$b_1$};
    
    \node[Aobj,below of=b1] (b01) {$\obj{b_0}{b_1}$};
    \node[Asol,below of=b01] (b01s) {};
    \node[Aproc,below of=b01s] (a1) {$a_1$};
    \node[Aobj,below of=a1] (a01) {$\obj{a_0}{a_1}$};
    \node[Asol,below of=a01] (a01s) {};
    
    
    \node[Aobj,below of=c2] (c12) {$\obj{c_1}{c_2}$};
    \node[Asol,below of=c12] (c12s) {};
    \node[Aproc,below of=c12s] (f0) {$f_0$};
    \node[Aobj,below of=f0] (f00) {$\obj{f_0}{f_0}$};
    \node[Asol,below of=f00] (f00s) {};
    
    %node label pour les probas
    \node[labelproba1, right of=d2,node distance=1.1cm] (pd2) {${\color{blue}\{\frac{13}{19}\}}$};
    \node[labelproba2, right of=d12,node distance=1.1cm] (pd12) {${\color{blue}\{\frac{13}{19}\}}$};
    \node[labelproba3, left of=b1,node distance=1.1cm] (pb1) {${\color{blue}\{1\}}$};
    \node[labelproba4, right of=c2,node distance=1.1cm] (pc2) {${\color{blue}\{\frac{2}{3}\}}$};
    \node[labelproba5, left of=b01,node distance=1.1cm] (pb01) {${\color{blue}\{1\}}$};
    \node[labelproba6, right of=c12,node distance=1.1cm] (pc12) {${\color{blue}\{\frac{2}{3}\}}$};
    \node[labelproba7, left of=a1,node distance=1.1cm] (pa1) {${\color{blue}\{1\}}$};
    \node[labelproba8, right of=f0,node distance=1.1cm] (pf0) {${\color{blue}\{1\}}$};
    \node[labelproba9, left of=a01,node distance=1.1cm] (pa01) {${\color{blue}\{1\}}$};
    \node[labelproba10, right of=f00,node distance=1.1cm] (pf00) {${\color{blue}\{1\}}$};
    \node[labelproba11, left of=a01s,node distance=1.1cm] (pa01s) {${\color{blue}\{1\}}$};
    \node[labelproba12, right of=f00s,node distance=1.1cm] (pf00s) {${\color{blue}\{1\}}$};
    
    \path
      (a2) edge (a02)
      (a02) edge (a02s1)
      (a02) edge (a02s2)
      (a02s1) edge (b1)
      (a02s2) edge (c2)
      (c2) edge (c12)
      (c12) edge (c12s)
      (c12s) edge (f0)
      (f0) edge (f00)
      (f00) edge (f00s)
      (b1) edge (b01)
      (b01) edge (b01s)
      (b01s) edge (a1)
      (a1) edge (a01)
      (a01) edge (a01s)
;
\end{tikzpicture}
}



