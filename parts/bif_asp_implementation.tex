\begin{frame}[c]
 \frametitle{ASP implementation}
 \framesubtitle{ASP modelling of Automata Networks}
 
\textbf{Declaration of local states, transitions, and states}
\begin{itemize}
\item \tval{Local states} : \lstinline|ls($a$,$i$)|.
\item \tval{Transitions \& conditions} : \lstinline|tr($id$,$a$,$i$,$j$)| and \lstinline|trcond($id$,$b$,$k$)|. 
($\antrl aij{\{b_k\}\cup\ell}\in\anT$)
\item \tval{States} : \lstinline|s(ID,A,I)|.
\item \tval{goal} : \lstinline|goal($g$,$1$)|.
\end{itemize}


\textbf{Example:}
\begin{columns}
\begin{column}{0.5\textwidth}
\begin{lstlisting}
ls(a,0). ls(a,1). ls(a,2).\\
tr(1,a,1,0).\\
tr(2,a,0,1). trcond(2,b,0).\\
tr(3,a,0,2). trcond(3,b,0). trcond(3,c,0).\\
s(0,a,0). s(0,b,0). s(0,c,0).        goal(a,2).
\end{lstlisting}
\end{column}
\begin{column}{0.5\textwidth}
\begin{center}\scalebox{\scaleex}{
\begin{tikzpicture}
\exanaspdef
\end{tikzpicture}
}\end{center}
\end{column}
\end{columns}
\end{frame}

\begin{frame}
\frametitle{ASP implementation}
 \framesubtitle{ASP modelling of Reachability}

\textbf{Declaration of $s_b$, $t_b$, and $s_u$} \\
\vspace{0.1in}
\begin{lstlisting}
$1 \{ btr(ID)$ : $tr(ID,\_,\_,\_) \} 1.$\\
$s(u,A,J)$ :- $btr(ID),tr(ID,A,\_,J).$\\
$s(u,B,K)$ :- $btr(ID),trcond(ID,B,K).$\\
$s(b,A,I)$ :- $btr(ID),tr(ID,A,I,\_).$\\
$s(b,B,K)$ :- $s(u,B,K),btr(ID),tr(ID,A,\_,\_),B!=A.$\\
\end{lstlisting}
\vspace{0.1in}
\textbf{\iIIIa declaration of $s_b\in\unf(s_0)$}\\
\vspace{0.1in}
\lstinline|reach($a$,$i$)|  is true if a
reachable state contains $a_i$.
Declaring $s_b$ reachable from $s_0$ is done simply as follows:\\
\vspace{0.1in}
\begin{lstlisting}
reach(A,I) :- s(b,A,I).
\end{lstlisting}
\vspace{0.1in}
\end{frame}

\begin{frame}
\frametitle{ASP implementation}
 \framesubtitle{ASP modelling of Reachability}

\textbf{\iI declaration of $\neg \OA{s_u}{g_1}$} \\
\vspace{0.1in}
\begin{lstlisting}
:- $oa\_valid(u,ls(G,I)),goal(G,I).$ \\
\end{lstlisting}
\vspace{0.1in}


\textbf{\iII declaration of $\UA{s_b}{g_1}$}\\
\vspace{0.1in}
\begin{lstlisting}
ua\_lcg(b,root,ls(G,I)) :- goal(G,I).\\
ctx(b,A,I) :- btr(ID),tr(ID,A,I,\_).\\
ctx(b,B,K) :- s(u,B,K),btr(ID),tr(ID,A,\_,\_),B!=A.\\
1 \{ s(b,A,I) : ctx(b,A,I) \} 1 :- ctx(b,A,\_).\\
\end{lstlisting}
\vspace{0.1in}

\textbf{\iIIIb declaration of $\UA{s_0}{s_b}$}\\
\vspace{0.1in}
\begin{lstlisting}
ua\_lcg(0,root,ls(A,I)) :- s(b,A,I).\\
ctx(0,A,I) :- s(0,A,I).\\
\end{lstlisting}
\vspace{0.1in}
\end{frame}



\begin{comment}
Therefore, our characterization is sound (no false positive) but incomplete: some $t_b$
might be missed (false negatives).
Using \iIIIa instead of \iIIIb potentially reduces the false negatives, at the condition that the
prefix of the unfolding is tractable. When facing a model too large for the unfolding approach, we
should rely on \iIIIb which is much more scalable but may lead to more false
negatives.

Relying on the unfolding from $s_b$ ($\unf(s_b)$) is not considered here, as it would require to
compute a prefix from each $s_b$ candidate, whereas $\unf(s_0)$ is computed only once before the
bifurcation identification.
\end{comment}
