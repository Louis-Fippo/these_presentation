\begin{frame}[c]
  \frametitle{Synthèse sur l'analyse statique des propriétés quantitatives}

%mots clés: large scale
%bien expliquer les résultats obtenus
% \textbf{Summary}

%\pause
\tval{Approximation des propriétés quantitatives}\\
\medskip
\textbf{Interprétation abstraite des scénarios}
  \begin{itemize}
   \item Causalité locale quantitative.
   \item Graphe de causalité locale quantifié.
  \end{itemize}
\medskip
\textbf{Outil d'aide à la décision}
\begin{itemize}
 \item Borne inférieure de la probabilité et des délais.
 \item Choix des scénarios (plus probable, plus rapide,...).
\end{itemize}
\medskip
\textbf{Intérêt:}
\begin{center}
 \begin{tikzpicture}
  \node[esdef] (defverif) at (4,-5) {\begin{tabular}{c} 
  Exprimer les formules comme: quelle est la probabilité d'observer \\
  une protéine donnée  $a$ après avoir observé une protéine $b$?
   \end{tabular}};
 \end{tikzpicture}

\end{center}



\end{frame}