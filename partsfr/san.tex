% Définition du Process Hitting + sortes coopératives

%san presentation
\begin{comment}
\begin{frame}
\frametitle{Biological networks modelling}
\begin{tikzpicture}
\node (brn) at (0,5) {\begin{tikzpicture}[auto,scale=0.9] 
                                                 \path[use as bounding box] (-0.7,-0.3) rectangle (2.5,2);

                                                              \node[qgre,scale=0.8] (a) at (0,2) {a};
                                                              \node[mod,scale=0.8] (i) at (1,1) {i};
                                                              \node[qgre,scale=0.8] (b) at (0,0) {b};
                                                              \node[qgre,scale=0.8] (c) at (2,1) {c};
                                                              \path
                                                                (a) edge[inh,scale=0.1] (i)
                                                                (b) edge[act,scale=0.1] (i)
                                                                (i) edge[st,scale=0.1]  (c);
                                                              
                                                 \end{tikzpicture}};
\node[scale=0.6] (an) at (7,5) {\begin{tikzpicture}
\exandef
\end{tikzpicture}};
\node[scale=0.6] (sg) at (2,1) {\input{figures/sg-example3.pgf}};
\path 
(brn) edge[->,line width=8pt, color=lightgray] (an)
(an) edge[->,line width=8pt, color=lightgray,bend left] (sg)
;

\end{tikzpicture}
\end{frame}
\end{comment}

\begin{frame}
  \frametitle{Stochastic Automata Networks}
  %\framesubtitle{\tcite{Paulev\'e et al. 2012}}
\begin{columns}
\begin{column}{0.6\textwidth}

% Automate sans les rates
\only<1->{
\tikzstyle{tick}=[densely dotted]
\tikzstyle{hit}=[->,>=angle 45]
\tikzstyle{selfhit}=[min distance=30pt,curve to]
\tikzstyle{bounce}=[densely dotted,>=stealth',->]
\tikzstyle{hlhit}=[very thick]
\begin{center}\scalebox{\scaleex}{
\begin{tikzpicture}
\exandef
\end{tikzpicture}
}\end{center}
}
\end{column}

\begin{column}{0.4\textwidth}
\begin{figure}[p]
\centering

\scalebox{0.4}{

\only<1->{
\tikzstyle{arc0}=[->]
\tikzstyle{nd0}=[]
\tikzstyle{arc1}=[->]
\tikzstyle{nd1}=[]
\tikzstyle{arc2}=[->]
\tikzstyle{nd2}=[]
\tikzstyle{arc3}=[->]
\tikzstyle{nd3}=[]
\tikzstyle{arc4}=[->]
\tikzstyle{nd4}=[]
\tikzstyle{arc5}=[->]
\tikzstyle{nd5}=[]
\input{figures/sg-example3.pgf}
}
}

\end{figure}

\end{column}
\end{columns}

\begin{liste}
  \item \tval{Automata}: components \qex{$a$, $b$, $c$}
  \item \tval{local states}: levels of expression \qex{$c_0$, $c_1$, $c_2$}
  \item \tval{States}: sets of active local states%
  \only<1->{\qex{$\PHetat{a_2, b_1, c_0}$}}
  \item \tval{Transitions}: dynamics \qex{\only<1->{$t_1 = \trans{a_0}{a_1}{b_0}$}, \only<1->{$t_2 = \trans{a_1}{a_0}{}$}, \only<1->{$t_3 = \trans{a_0}{a_2}{b_0,c_0}$}, \only<1->{$t_4 = \trans{b_0}{b_1}{}$}}
\end{liste}
\end{frame}

\begin{frame}
  \frametitle{Stochastic Automata Networks}
  %\framesubtitle{\tcite{Paulev\'e et al. 2012}}
\begin{columns}
\begin{column}{0.6\textwidth}

% Automate avec les rates
\only<1->{
\tikzstyle{tick}=[densely dotted]
\tikzstyle{hit}=[->,>=angle 45]
\tikzstyle{selfhit}=[min distance=30pt,curve to]
\tikzstyle{bounce}=[densely dotted,>=stealth',->]
\tikzstyle{hlhit}=[very thick]
\begin{center}\scalebox{\scaleex}{
\begin{tikzpicture}
\exsandef
\end{tikzpicture}
}\end{center}
}
\end{column}

\begin{column}{0.4\textwidth}
\begin{figure}[p]
\centering

\scalebox{0.4}{

\only<1->{
\tikzstyle{arc0}=[->]
\tikzstyle{nd0}=[]
\tikzstyle{arc1}=[->]
\tikzstyle{nd1}=[]
\tikzstyle{arc2}=[->]
\tikzstyle{nd2}=[]
\tikzstyle{arc3}=[->]
\tikzstyle{nd3}=[]
\tikzstyle{arc4}=[->]
\tikzstyle{nd4}=[]
\tikzstyle{arc5}=[->]
\tikzstyle{nd5}=[]
\input{figures/sg-example3-rate.pgf}
}
}

\end{figure}

\end{column}
\end{columns}

\begin{liste}
  \item \tval{Automata}: components \qex{$a$, $b$, $c$}
  \item \tval{local states}: levels of expression \qex{$c_0$, $c_1$, $c_2$}
  \item \tval{States}: sets of active local states%
  \only<1->{\qex{$\PHetat{a_2, b_1, c_0}$}}
  \item \tval{Transitions}: dynamics \qex{\only<1->{$t_1 = \trans{a_0}{a_1}{b_0,\mathbf{\color{Maroon} 2}}$}, \only<1->{$t_2 = \trans{a_1}{a_0}{\mathbf{\color{Maroon} 1}}$}, \only<1->{$t_3 = \trans{a_0}{a_2}{b_0,c_0,\mathbf{\color{Maroon} 2}}$}, \only<1->{$t_4 = \trans{b_0}{b_1}{\mathbf{\color{Maroon} 3}}$}}
\end{liste}
\end{frame}

\begin{frame}
\frametitle{SAN as CTMC}
\begin{figure}[p]
\centering
\scalebox{0.4}{
\input{figures/sg-example2-rate.pgf}
}

\label{ex-bifurcations}
\end{figure}
Bad news: \textcolor{red}{state space explosion}
\end{frame}

\begin{frame}
\frametitle{SAN as CTMC}
\begin{columns}
 \begin{column}{0.5\textwidth}
 \begin{figure}[p]
\centering
\scalebox{0.25}{
\input{figures/sg-example2-rate.pgf}
}

\label{ex-bifurcations}
\end{figure}
Bad news: \textcolor{red}{state space explosion}
  
 \end{column}
 
 \begin{column}{0.5\textwidth}
  \textbf{Probability to have a transition from state s:}
  $$\ppexit{s}{*}{t} = 1-e^{-\lambda t}$$ \\
  \medskip
  \textbf{Probability to have a transition from s to s':}
  $$\ppexit{s}{s^{'}}{t} = \frac{R (s,s^{'})}{E (s)} \cdot \ppexit{s}{*}{t}$$\\
  \medskip
  \textbf{Probability to have a transition from s to a set of states:}
  $$\ppexit{s}{A}{t} = \frac{R (s,A)}{E (s)} \cdot \ppexit{s}{*}{t}$$ \\
 \end{column}

\end{columns}

\end{frame}


