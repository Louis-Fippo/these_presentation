
\begin{frame}[c]
 \frametitle{Modélisation hybride}
 %\framesubtitle{Concept}
  %\pause
 %figure illustrative
 \begin{tikzpicture}[auto]
  
\path[use as bounding box] (-0.7,-2) rectangle (3,3);

%le noeud pour les connaissances de la littérature, générales
\node[align=center] (gk) at (2,3) {\begin{tabular}{|c|} 
\hline
 Connaissances générales  \\
 \hline
 Littérature  \\
  \hline
 Hypothèses   \\
  \hline
\end{tabular}};

%\pause

%les noeud pour le réseau biologique
\node[qgre] (a) at (1,1.5) {a};
\node[mod] (i) at (2,1) {i};
\node[qgre] (b) at (1,0.5) {b};
\node[qgre] (c) at (2.5,1) {c};


\path
 (a) edge[act] (i)
 (b) edge[inh] (i)
 (i) edge[st]  (c);
 
 %\pause

\node (deco) at (2,-0.1) {Données de séries temporelles};
\node[align=center] (tsd) at (2,-1) {\begin{tabular}{|c|c|c|c|} 
\hline
 Genes  & 1h & ... & 24h  \\
 \hline
 Gene $1$  &   & ...  &    \\
  \hline
  Gene $2$  &   & ...  &    \\
  \hline
\end{tabular}};


\onslide<2->{

\node (d1) at (5,1) {};
\node (d2) at (7.5,1) {};

\node (d3) at (4.5,-1.5) {};
\node (d4) at (4.5,3.5) {};


\draw[->,line width=6pt, color=lightgray] (d1) -- (d2) node[above=10pt,midway]{\textcolor{black}{\textbf{Modélisation algébrique}}};
}



%\draw[decoration={brace,amplitude=12pt}, 
%decorate,line width=2pt,gray] (d3) -- (d4) node[above=10pt,midway]{\textcolor{black}{\textbf{}}};


\onslide<2->{
%le modèle en process hitting
\node[scale=0.4] (phmodel) at (10,1) {\begin{tikzpicture} \exphHM 
                           \end{tikzpicture}};
}

\end{tikzpicture}

\onslide<2->{
 \begin{block}
  
  \begin{itemize}
   \item Caractérisation formelle de la topologie et de la dynamique des RRB.
   \item Intégration des données de séries temporelles.
   \item Raffinement qualitatif et quantitatif de la dynamique.
   \item Simulation stochastique et analyse statistique des traces.
  \end{itemize}

 \end{block}
}

\end{frame}
