%%explain the automatic model construction

\begin{frame}[c]
\frametitle{Network topology}
\framesubtitle{Automatic patterns detection and transformation}

%We have two procedures that allow us to detect and translate RSTC network to the Process Hitting model.

\begin{tikzpicture}[auto]
 
 % placement des noeuds
 
 %\node[cloud, cloud puffs = 10, draw, minimum width = 0.1cm, minimum height = 0.1cm, fill = gray!10, scale=0.3] (rstcbis) at (-2,13){
 %  \begin{tabular}{c}
 %  \textbf{Biological network} \\ \hline
 %  RSTC network
 % \end{tabular}};
 
 \node (rep) at (0,-5) {};
  
  \node[es,align=center,scale=0.5] (rstcbis) at (-2,11.5) {\begin{tabular}{c}
                                            \textbf{Biological network} \\ \hline
                                            RSTC network
                                           \end{tabular}};

 
\onslide<2->{
%\draw[-open triangle 90] (0,12.5) -- (0.2,12.5);
 \node[instruct,align=center, scale=0.7] (pro1) at (0.5,11.5) {\begin{tabular}{c}
                                            \textbf{Patterns Detection} \\ \hline
                                            in  RSTC network
                                          \end{tabular}};
      
 \path
  (rstcbis) edge[st] (pro1);
}
                                          
 \onslide<3->{
 %\draw[-open triangle 90] (2,12.5) -- (2.5,12.5);
 \node[es, scale=0.5] (sop) at (3,11.5) {\begin{tabular}{c}
                                            \textbf{Set of patterns} \\ \hline
                                            of RSTC network
                                           \end{tabular}};

\path
   (pro1) edge[st] (sop);
                                           
\node[scale=0.8] (bpat) at (-1,8) {\begin{tabular}{|c|}
\hline
\textbf{Biological Patterns}

\\ \hline
\begin{tikzpicture}
\node[scale=0.7] (sa1) at (0,0){\begin{tikzpicture}[auto]
\path[use as bounding box] (-0.7,-0.3) rectangle (2.5,2);

\node[qgre] (a) at (0,0.5) {a};
\node[mod] (i) at (1,0.5) {i};
\node[qgre] (b) at (2,0.5) {b};
\node[es] (d) at (1,1.5) {Simple activation};

% a restorer
\path
 (a) edge[act] (i)
 (i) edge[st]  (b);
\end{tikzpicture}};
\end{tikzpicture}

\\ \hline

\begin{tikzpicture}
\node[scale=0.7] (si1) at (0,0){\begin{tikzpicture}[auto]
\path[use as bounding box] (-0.7,-0.3) rectangle (2.5,2);

\node[qgre] (a) at (0,0.5) {a};
\node[mod] (i) at (1,0.5) {i};
\node[qgre] (b) at (2,0.5) {b};
\node[es] (d) at (1,1.5) {Simple inhibition};

%a restorer 
\path
 (a) edge[inh] (i)
 (i) edge[st]  (b);
 \end{tikzpicture}};
\end{tikzpicture}


\\ \hline

\begin{tikzpicture}
\node[scale=0.7] (sai1) at (0,0){\begin{tikzpicture}[auto]
\path[use as bounding box] (-0.7,-0.3) rectangle (2.5,3);
\node[qgre] (a) at (0,2) {a};
\node[mod] (i) at (1,1) {i};
\node[qgre] (b) at (0,0) {b};
\node[qgre] (c) at (2,1) {c};
\node[es] (d) at (1,3) {activation or inhibition};

% arestorer
\path
 (a) edge[act] (i)
 (b) edge[inh] (i)
 (i) edge[st]  (c);
\end{tikzpicture}};
\end{tikzpicture}

\\ \hline

\end{tabular}
};

}


\onslide<4->{

\node[es,scale=0.5] (timed) at (3,10.5) {\begin{tabular}{c}
                                            \textbf{Time delay} \\ \hline
                                            Estimate from TSD
                                           \end{tabular}};


 \node[instruct,align=center, scale=0.7] (pro2) at (5.5,11.5) {\begin{tabular}{c}
                                            \textbf{Patterns translation} \\ \hline
                                            in  AN (PH) model
                                           \end{tabular}};
 \path
    (sop) edge[st] (pro2)
    (timed) edge[st] (pro2);
 }
 
 \onslide<5->{
   
   \node[es,scale=0.5] (php) at (8,11.5) {\begin{tabular}{c}
                                            \textbf{AN (PH) Patterns} \\
                                           \end{tabular}};
\path
   (pro2) edge[st] (php);

\node[scale=0.8] (phpat) at (6.8,8) {\begin{tabular}{|c|c|}
\hline

\textbf{AN (Process Hitting) Transformations}

 \\ \hline

\begin{tikzpicture}
%\exphpatact
\node[scale=0.5] (sa) at (0,0) {\begin{tikzpicture}
\exphpatact
\end{tikzpicture}};
\end{tikzpicture}


\\ \hline


\begin{tikzpicture}
%\exphpatact
\node[scale=0.5] (sa) at (0,0) {\begin{tikzpicture}
\exphpatini
\end{tikzpicture}};
\end{tikzpicture}

 \\ \hline


\begin{tikzpicture}
%\exphpatact
\node[scale=0.5] (sai) at (0,0) {\begin{tikzpicture}
\exphpatai
\end{tikzpicture}};
\end{tikzpicture}
\\ \hline

\end{tabular}
};
}
\end{tikzpicture}

%exemple of patterns 


\end{frame}


\begin{frame}[c]
 \frametitle{Network topology}
 \framesubtitle{Automatic patterns detection (method)} 

\begin{columns}
\begin{column}{0.6\textwidth}

\begin{tikzpicture}[auto]

\node[qgre] (a1) at (-3,-3) {a};
\node[qgre] (a2) at (-1,-3) {b};
\node[qgre] (a3) at (1,-3) {c};
\node[qgre] (a4) at (3,-3) {d};
%\node[qgre] (a5) at (5,-3) {e};

\node[mod] (i1) at (-3,-2) {};
\node[mod] (i2) at (-1,-2) {};
\node[mod] (i3) at (1,-2) {};
\node[mod] (i4) at (3,-2) {};
%\node[mod] (i5) at (5,-2) {};

\node[ps] (ps1) at (-3,-1) {PS1};
\node[ps] (ps2) at (-1,-1) {PS2};
\node[ps] (ps3) at (1,-1) {PS3};
\node[ps] (ps4) at (3,-1) {PS4};
%\node[ps] (ps5) at (5,-1) {PS5};

\node[mod] (i6) at (-3,0) {};
%\node[mod] (i7) at (-1,0) {};
\node[mod] (i8) at (1,0) {};
\node[mod] (i9) at (3,0) {};
%\node[mod] (i10) at (5,0) {};

\node[ps] (ps6) at (-3,1) {PS6};
%\node[ps] (ps7) at (-1,1) {PS7};
\node[ps] (ps8) at (1,1) {PS8};
\node[ps] (ps9) at (3,1) {PS9};
%\node[ps] (ps10) at (5,1) {PS10};

%les complex
\node[ecad] (d) at (0,3) {Ecad};
\node[cplx] (c1) at (2,-1) {cplx1};
\node[cplx] (c2) at (0,1) {cplx2};
\node[cplx] (c3) at (-2,-1) {cplx3};
\node[cplx] (c4) at (-2,1) {cplx4};
\node[cplx] (c5) at (0,0) {cplx5};
\node[cplx] (c6) at (2,1) {cplx6};

%les seed node
\node[sn] (sn1) at (-2,-3) {SN1};
\node[sn] (sn2) at (2,-3) {SN2};


%les edges 

\path
 (i1) edge[st] (a1)
 (i2) edge[st] (a2)
 (i3) edge[st]  (a3)
 (i4) edge[st]  (a4)
 %(i5) edge[st]  (a5)
 
 (ps1) edge[act] (i1)
 (ps2) edge[act] (i2)
 (ps3) edge[act] (i2)
 (ps3) edge[act] (i3)
 (ps4) edge[act] (i4)
 %(ps5) edge[act] (i5)
 
 (i6) edge[st] (ps1)
 (i6) edge[st] (ps2)
 (i8) edge[st] (ps3)
 (i9) edge[st] (ps4)
 (i9) edge[st] (ps4)
 %(i9) edge[st] (ps5)
 %(i10) edge[st] (ps5)
 
 (ps6) edge[act] (i6)
 (ps8) edge[act] (i8)
 (ps9) edge[act] (i9)
 %(ps10) edge[act] (i10)
 
 (d) edge[act] (ps6)
 (d) edge[act] (ps8)
 (d) edge[act] (ps9)
 %(d) edge[act] (ps10)
 
 (c4) edge[act] (c5)
 (c2) edge[act] (c5)
 (d) edge[act] (c2)
 (c5) edge[inh] (ps2)
 (c6) edge[act] (i8)
 (c6) edge[act] (i9)
 (i8) edge[st] (c1)
 (i9) edge[st] (c1)
 (c1) edge[act] (i4)
 
 (ps3) edge[inh] (sn2)
 (i4) edge[st] (sn2)
 
 (ps1) edge[inh] (sn1)
 (c3) edge[act] (sn1)
 (c4) edge[act] (c3)
 
 (a1) edge[inh, bend left] (ps6);
 

 %pattern 1
\only<2-5>{
\node[snpat] (snpat1) at (1,-3) {};
}
\only<3-5>{
\path
 (a3) edge[stv] (i3);
 }
 \only<4-5>{
\path
 (i3) edge[stv] (ps3);
 }
  \only<5>{
\node[snpat] (snpat2) at (1,-1) {};
}

  %le pattern rajouté
\only<5-9>{
\node[scale=0.9] (patraj1) at (5,-2){\begin{tikzpicture}[auto]
\path[use as bounding box] (-0.7,-0.3) rectangle (2.5,2);

\node[ps] (aps3) at (0,0.5) {PS3};
\node[mod] (ips3) at (1,0.5) {i};
\node[qgre] (cps3) at (2,0.5) {c};

% a restorer
\path
 (aps3) edge[act] (ips3)
 (ips3) edge[st]  (cps3);
\end{tikzpicture}};

}

%pattern 2

%le pattern rajouté
\only<9>{
\node[scale=0.9] (sai1) at (5,-3.5){\begin{tikzpicture}[auto]
\path[use as bounding box] (-0.7,-0.3) rectangle (2.5,3);
\node[ps] (aps2) at (0,2) {PS2};
\node[mod] (ips2) at (1,1) {i};
\node[ps] (bps3) at (0,0) {PS3};
\node[qgre] (cb) at (2,1) {b};

% arestorer
\path
 (aps2) edge[act] (ips2)
 (bps3) edge[act] (ips2)
 (ips2) edge[st]  (cb);
\end{tikzpicture}};


}

 \only<6-9>{
\node[snpat] (snpat3) at (-1,-3) {};
}


 \only<7-9>{
\path
 (a2) edge[stv] (i2);
 }
 \only<8-9>{
\path
 (i2) edge[stv] (ps2)
 (i2) edge[stv] (ps3);
 }
 
  \only<9>{
\node[snpat] (snpat4) at (-1,-1) {};
\node[snpat] (snpat5) at (1,-1) {};
}


 
\end{tikzpicture}

\end{column}

\begin{column}{0.4\textwidth}
Two main types of nodes:
 \begin{liste}
  \item \tval{Terminal Nodes:} proteins, complexes, mRNA expressions, cellular states,...
  \item \tval{Transient Nodes:} translocations, modifications, transcriptions,...
 \end{liste}
 
 \only<10->{
 \tval{Complexity}:
 The algorithm  has a time complexity of \tval{$\mathcal{O}(|V|\log{}(h))$}. 
  Where 
  \begin{liste}
  \item \tval{$|V|$} is the set of nodes and \tval{$|E|$} is the set of edges.
  \item $h$ is the average height of the patterns in the RSTC network. In the worst case 
  $h = \log_{|V|}(|V|)$.
 \end{liste}
 }
\end{column}



\end{columns}

\end{frame}

