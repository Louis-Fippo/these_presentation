
\begin{frame}[c]
\frametitle{Relaxation du problème de l'identification des bifurcations}

\begin{figure}[t]
%\centering
\scalebox{0.6}{
\begin{tikzpicture}[node distance=1.5cm,left]
\node (s0) [circle,fill=gray!60] {$s_0$};
\node (sb) [circle,fill=gray!60,above right of=s0,xshift=2cm] {$s_b$};
\node (g) [circle,fill=blue!20,minimum width=7mm,below right of=sb,xshift=2cm] {$g_1$};
\node (su) [above right of=sb] {$s_u$};
\node (u) [below right of=su,xshift=5mm] {};
\path[->] (sb) edge[->,red,thick] node[auto] {$t_b$} (su) ;
\path[->,bend left=20,densely dashed]
        (s0) edge (sb)
        (sb) edge (g)
        (s0) edge [bend right=10] (g)
        (su) edge[>={Rays[length=3mm,width=3mm,red]}] (u)
;
\end{tikzpicture}
}
\end{figure}

{\color{magenta}{\#}}: conditions suffisantes.
\pause
\begin{align*}
\text{\cI}\enspace &s_b\play t_b\nreach g_1
&
\text{\iI}\enspace&\neg \OA{s_u}{g_1}
&
\text{\iI}&\Rightarrow\text{\cI}
\end{align*}

%rajouter sufficient condition/ précisier ce que veut dire sharp
%remplacer unf-prefix par reach.
\pause
\begin{align*}
\text{\cII}\enspace &s_b\reach g_1
&
\text{\iII}\enspace&\UA{s_b}{g_1}
&
\text{\iII}&\Rightarrow\text{\cII}
\end{align*}

\pause
\begin{align*}
\text{\cIII}\enspace &s_0\reach s_b
&
\begin{split}
\text{\iIIIa}\enspace&s_b\in\freach(s_0)\\
\text{\iIIIb}\enspace&\UA{s_0}{s_b}
\end{split}
&
\begin{split}
\text{\iIIIa}&\Leftrightarrow\text{\cIII}\\
\text{\iIIIb}&\Rightarrow\text{\cIII}
\end{split}
\end{align*}

\pause
\begin{align*}
\text{\iI} \ et\ \text{\iII}\ et \ (\text{\iIIIa} ou \text{\iIIIb}) \ \Rightarrow t_b\ est\ une\ bifurcation.
\end{align*}
\end{frame}



\begin{comment}
\begin{frame}[c]
 \frametitle{General scheme for identification of bifurcation}
%\framesubtitle{concepts and tools}
 
%A sound and complete characterization of the local transitions $t_b\in\anT$ triggering a bifurcation
%from state $s_0$ to the goal $g_1$ would be the following:
%$t_b$ is a bifurcation transition if and only if there exists a state $s_b\in\anS$ such that
\begin{align*}
\text{\cI}\enspace &s_b\play t_b\nreach g_1
&
\text{\cII}\enspace &s_b\reach g_1
&
\text{\cIII}\enspace &s_0\reach s_b
\end{align*}
%where $s_u = s_b\play t_b$, $s \nreach g_1 \EQDEF \forall s'\in\anS, s \reach s' \Rightarrow \get
%{s'}g\neq g_1$
%and $s\reach g_1\EQDEF \exists s_g\in\anS: \get{s_g}g = g_1\wedge s\reach s_g$.

%rajouter sufficient condition/ précisier ce que veut dire sharp
%remplacer unf-prefix par reach.

\begin{align*}
\text{\iI}\enspace&\neg \OA{s_u}{g_1}&
\text{\iII}\enspace&\UA{s_b}{g_1}&
\begin{split}
\text{\iIIIa}\enspace&s_b\in\unf(s_0)\\
\text{\iIIIb}\enspace&\UA{s_0}{s_b}
\end{split}
\end{align*}

%where $\unf(s_0)$ is the set of all reachable states from $s_0$ represented as the prefix of the
%unfolding of the AN which has to be pre-computed (\secref{unf}).
%Either \iIIIa or \iIIIb can be used, at discretion.
%From $\fOA$ and $\fUA$ properties (\secref{approx}), we directly obtain:
\begin{align*}
\text{\iI}&\Rightarrow\text{\cI}&
\text{\iII}&\Rightarrow\text{\cII}&
\begin{split}
\text{\iIIIa}&\Leftrightarrow\text{\cIII}\\
\text{\iIIIb}&\Rightarrow\text{\cIII}
\end{split}
\end{align*}

\begin{align*}
\text{\iI} \ and\ \text{\iII}\ and \ (\text{\iIIIa} or \text{\iIIIb}) \ \Rightarrow t_b\ is\ a\ bifurcation.
\end{align*}

\end{frame}
\end{comment}


