% Définition du Process Hitting + sortes coopératives

\begin{frame}
  \frametitle{Réseaux d'automates (AN)}
  %\framesubtitle{\tcite{Paulev\'e et al. 2012}}


\begin{columns}
\begin{column}{0.6\textwidth}
% 1 : Sortes
\only<1>{
\tikzstyle{process}=[circle,minimum size=15pt,font=\footnotesize,inner sep=1pt]
\tikzstyle{tick label}=[color=white, font=\footnotesize]
\tikzstyle{tick}=[transparent]
\tikzstyle{hit}=[transparent]
\tikzstyle{selfhit}=[transparent, min distance=30pt,curve to]
\tikzstyle{bounce}=[transparent]
\tikzstyle{hlhit}=[transparent]
\tikzstyle{local transitions}=[transparent]
\begin{center}\scalebox{\scaleex}{
\begin{tikzpicture}
\exandef
\end{tikzpicture}
}\end{center}
}

% 2 : Processus
\only<2>{
\tikzstyle{process}=[circle,draw,minimum size=15pt,font=\footnotesize,inner sep=1pt]
\tikzstyle{tick label}=[font=\footnotesize]
\tikzstyle{tick}=[densely dotted]
\tikzstyle{hit}=[transparent]
\tikzstyle{selfhit}=[transparent, min distance=30pt,curve to]
\tikzstyle{bounce}=[transparent]
\tikzstyle{hlhit}=[transparent]
\tikzstyle{local transitions}=[transparent]
\begin{center}\scalebox{\scaleex}{
\begin{tikzpicture}
\exandef
\end{tikzpicture}
}\end{center}
}

% 3 : États
\only<3>{
\tikzstyle{hit}=[transparent]
\tikzstyle{selfhit}=[transparent, min distance=30pt,curve to]
\tikzstyle{bounce}=[transparent]
\tikzstyle{hlhit}=[transparent]
\tikzstyle{local transitions}=[transparent]
\begin{center}\scalebox{\scaleex}{
\begin{tikzpicture}
\exandef

\TState{3}{a_0,b_0,c_0}
\end{tikzpicture}
}\end{center}
}

% 4 : Actions
\only<4->{
\tikzstyle{tick}=[densely dotted]
\tikzstyle{hit}=[->,>=angle 45]
\tikzstyle{selfhit}=[min distance=30pt,curve to]
\tikzstyle{bounce}=[densely dotted,>=stealth',->]
\tikzstyle{hlhit}=[very thick]
\begin{center}\scalebox{\scaleex}{
\begin{tikzpicture}
\exandef
\TState{4}{a_0,b_0,c_0}
\TState{5}{a_1,b_0,c_0}
\TState{6}{a_0,b_0,c_0}
\TState{7}{a_2,b_0,c_0}
\TState{8}{a_2,b_1,c_0}
\end{tikzpicture}
}\end{center}
}
\end{column}

\begin{column}{0.4\textwidth}
\begin{figure}[p]
\centering

\scalebox{0.4}{
\only<4>{
\tikzstyle{arc0}=[transparent]
\tikzstyle{nd0}=[]
\tikzstyle{arc1}=[transparent]
\tikzstyle{nd1}=[transparent]
\tikzstyle{arc2}=[transparent]
\tikzstyle{nd2}=[transparent]
\tikzstyle{arc3}=[transparent]
\tikzstyle{nd3}=[transparent]
\tikzstyle{arc4}=[transparent]
\tikzstyle{nd4}=[transparent]
\tikzstyle{arc5}=[transparent]
\tikzstyle{nd5}=[transparent]

\begin{tikzpicture}[line join=bevel,font=\LARGE]
%%
  \node (6) at (405.0bp,18.0bp) [reach,nd4] {$\langle a_2,b_1,c_0\rangle$};
  \node (2) at (405.0bp,90.0bp) [reach,nd3] {$\langle a_2,b_0,c_0\rangle$};
  \node (1) at (570.0bp,90.0bp) [reach,nd1] {$\langle a_1,b_0,c_0\rangle$};
  \node (0) at (469.0bp,162.0bp) [reach,nd0] {$s_0 = \langle a_0,b_0,c_0\rangle$};
  \node (4) at (469.0bp,234.0bp) [reach,nd5] {$\langle a_0,b_1,c_0\rangle$};
 
 
  \draw [->,arc3] (0) ..controls (445.65bp,135.46bp) and (435.95bp,124.86bp)  .. (2);
  \draw [->,arc5] (0) ..controls (475.71bp,188.03bp) and (475.94bp,197.36bp)  .. (4);
  \draw [->,arc4] (2) ..controls (405.0bp,63.983bp) and (405.0bp,54.712bp)  .. (6);
  \draw [->,arc1] (0) ..controls (499.93bp,134.79bp) and (517.41bp,122.58bp)  .. (1);
  \draw [->,arc2] (1) ..controls (539.0bp,117.26bp) and (521.52bp,129.47bp)  .. (0);
  \draw [->,arc5] (4) ..controls (462.3bp,208.35bp) and (462.06bp,199.03bp)  .. (0);
 
%
\end{tikzpicture}


}
\only<5>{
\tikzstyle{arc0}=[->]
\tikzstyle{nd0}=[]
\tikzstyle{arc1}=[->]
\tikzstyle{nd1}=[]
\tikzstyle{arc2}=[transparent]
\tikzstyle{nd2}=[transparent]
\tikzstyle{arc3}=[transparent]
\tikzstyle{nd3}=[transparent]
\tikzstyle{arc4}=[transparent]
\tikzstyle{nd4}=[transparent]
\tikzstyle{arc5}=[transparent]
\tikzstyle{nd5}=[transparent]

\begin{tikzpicture}[line join=bevel,font=\LARGE]
%%
  \node (6) at (405.0bp,18.0bp) [reach,nd4] {$\langle a_2,b_1,c_0\rangle$};
  \node (2) at (405.0bp,90.0bp) [reach,nd3] {$\langle a_2,b_0,c_0\rangle$};
  \node (1) at (570.0bp,90.0bp) [reach,nd1] {$\langle a_1,b_0,c_0\rangle$};
  \node (0) at (469.0bp,162.0bp) [reach,nd0] {$s_0 = \langle a_0,b_0,c_0\rangle$};
  \node (4) at (469.0bp,234.0bp) [reach,nd5] {$\langle a_0,b_1,c_0\rangle$};
 
 
  \draw [->,arc3] (0) ..controls (445.65bp,135.46bp) and (435.95bp,124.86bp)  .. (2);
  \draw [->,arc5] (0) ..controls (475.71bp,188.03bp) and (475.94bp,197.36bp)  .. (4);
  \draw [->,arc4] (2) ..controls (405.0bp,63.983bp) and (405.0bp,54.712bp)  .. (6);
  \draw [->,arc1] (0) ..controls (499.93bp,134.79bp) and (517.41bp,122.58bp)  .. (1);
  \draw [->,arc2] (1) ..controls (539.0bp,117.26bp) and (521.52bp,129.47bp)  .. (0);
  \draw [->,arc5] (4) ..controls (462.3bp,208.35bp) and (462.06bp,199.03bp)  .. (0);
 
%
\end{tikzpicture}


}
\only<6>{
\tikzstyle{arc0}=[->]
\tikzstyle{nd0}=[]
\tikzstyle{arc1}=[->]
\tikzstyle{nd1}=[]
\tikzstyle{arc2}=[->]
\tikzstyle{nd2}=[]
\tikzstyle{arc3}=[transparent]
\tikzstyle{nd3}=[transparent]
\tikzstyle{arc4}=[transparent]
\tikzstyle{nd4}=[transparent]
\tikzstyle{arc5}=[transparent]
\tikzstyle{nd5}=[transparent]

\begin{tikzpicture}[line join=bevel,font=\LARGE]
%%
  \node (6) at (405.0bp,18.0bp) [reach,nd4] {$\langle a_2,b_1,c_0\rangle$};
  \node (2) at (405.0bp,90.0bp) [reach,nd3] {$\langle a_2,b_0,c_0\rangle$};
  \node (1) at (570.0bp,90.0bp) [reach,nd1] {$\langle a_1,b_0,c_0\rangle$};
  \node (0) at (469.0bp,162.0bp) [reach,nd0] {$s_0 = \langle a_0,b_0,c_0\rangle$};
  \node (4) at (469.0bp,234.0bp) [reach,nd5] {$\langle a_0,b_1,c_0\rangle$};
 
 
  \draw [->,arc3] (0) ..controls (445.65bp,135.46bp) and (435.95bp,124.86bp)  .. (2);
  \draw [->,arc5] (0) ..controls (475.71bp,188.03bp) and (475.94bp,197.36bp)  .. (4);
  \draw [->,arc4] (2) ..controls (405.0bp,63.983bp) and (405.0bp,54.712bp)  .. (6);
  \draw [->,arc1] (0) ..controls (499.93bp,134.79bp) and (517.41bp,122.58bp)  .. (1);
  \draw [->,arc2] (1) ..controls (539.0bp,117.26bp) and (521.52bp,129.47bp)  .. (0);
  \draw [->,arc5] (4) ..controls (462.3bp,208.35bp) and (462.06bp,199.03bp)  .. (0);
 
%
\end{tikzpicture}


}
\only<7>{
\tikzstyle{arc0}=[->]
\tikzstyle{nd0}=[]
\tikzstyle{arc1}=[->]
\tikzstyle{nd1}=[]
\tikzstyle{arc2}=[->]
\tikzstyle{nd2}=[]
\tikzstyle{arc3}=[->]
\tikzstyle{nd3}=[]
\tikzstyle{arc4}=[transparent]
\tikzstyle{nd4}=[transparent]
\tikzstyle{arc5}=[transparent]
\tikzstyle{nd5}=[transparent]

\begin{tikzpicture}[line join=bevel,font=\LARGE]
%%
  \node (6) at (405.0bp,18.0bp) [reach,nd4] {$\langle a_2,b_1,c_0\rangle$};
  \node (2) at (405.0bp,90.0bp) [reach,nd3] {$\langle a_2,b_0,c_0\rangle$};
  \node (1) at (570.0bp,90.0bp) [reach,nd1] {$\langle a_1,b_0,c_0\rangle$};
  \node (0) at (469.0bp,162.0bp) [reach,nd0] {$s_0 = \langle a_0,b_0,c_0\rangle$};
  \node (4) at (469.0bp,234.0bp) [reach,nd5] {$\langle a_0,b_1,c_0\rangle$};
 
 
  \draw [->,arc3] (0) ..controls (445.65bp,135.46bp) and (435.95bp,124.86bp)  .. (2);
  \draw [->,arc5] (0) ..controls (475.71bp,188.03bp) and (475.94bp,197.36bp)  .. (4);
  \draw [->,arc4] (2) ..controls (405.0bp,63.983bp) and (405.0bp,54.712bp)  .. (6);
  \draw [->,arc1] (0) ..controls (499.93bp,134.79bp) and (517.41bp,122.58bp)  .. (1);
  \draw [->,arc2] (1) ..controls (539.0bp,117.26bp) and (521.52bp,129.47bp)  .. (0);
  \draw [->,arc5] (4) ..controls (462.3bp,208.35bp) and (462.06bp,199.03bp)  .. (0);
 
%
\end{tikzpicture}


}
\only<8>{
\tikzstyle{arc0}=[->]
\tikzstyle{nd0}=[]
\tikzstyle{arc1}=[->]
\tikzstyle{nd1}=[]
\tikzstyle{arc2}=[->]
\tikzstyle{nd2}=[]
\tikzstyle{arc3}=[->]
\tikzstyle{nd3}=[]
\tikzstyle{arc4}=[->]
\tikzstyle{nd4}=[]
\tikzstyle{arc5}=[transparent]
\tikzstyle{nd5}=[transparent]

\begin{tikzpicture}[line join=bevel,font=\LARGE]
%%
  \node (6) at (405.0bp,18.0bp) [reach,nd4] {$\langle a_2,b_1,c_0\rangle$};
  \node (2) at (405.0bp,90.0bp) [reach,nd3] {$\langle a_2,b_0,c_0\rangle$};
  \node (1) at (570.0bp,90.0bp) [reach,nd1] {$\langle a_1,b_0,c_0\rangle$};
  \node (0) at (469.0bp,162.0bp) [reach,nd0] {$s_0 = \langle a_0,b_0,c_0\rangle$};
  \node (4) at (469.0bp,234.0bp) [reach,nd5] {$\langle a_0,b_1,c_0\rangle$};
 
 
  \draw [->,arc3] (0) ..controls (445.65bp,135.46bp) and (435.95bp,124.86bp)  .. (2);
  \draw [->,arc5] (0) ..controls (475.71bp,188.03bp) and (475.94bp,197.36bp)  .. (4);
  \draw [->,arc4] (2) ..controls (405.0bp,63.983bp) and (405.0bp,54.712bp)  .. (6);
  \draw [->,arc1] (0) ..controls (499.93bp,134.79bp) and (517.41bp,122.58bp)  .. (1);
  \draw [->,arc2] (1) ..controls (539.0bp,117.26bp) and (521.52bp,129.47bp)  .. (0);
  \draw [->,arc5] (4) ..controls (462.3bp,208.35bp) and (462.06bp,199.03bp)  .. (0);
 
%
\end{tikzpicture}


}
\only<9>{
\tikzstyle{arc0}=[->]
\tikzstyle{nd0}=[]
\tikzstyle{arc1}=[->]
\tikzstyle{nd1}=[]
\tikzstyle{arc2}=[->]
\tikzstyle{nd2}=[]
\tikzstyle{arc3}=[->]
\tikzstyle{nd3}=[]
\tikzstyle{arc4}=[->]
\tikzstyle{nd4}=[]
\tikzstyle{arc5}=[->]
\tikzstyle{nd5}=[]

\begin{tikzpicture}[line join=bevel,font=\LARGE]
%%
  \node (6) at (405.0bp,18.0bp) [reach,nd4] {$\langle a_2,b_1,c_0\rangle$};
  \node (2) at (405.0bp,90.0bp) [reach,nd3] {$\langle a_2,b_0,c_0\rangle$};
  \node (1) at (570.0bp,90.0bp) [reach,nd1] {$\langle a_1,b_0,c_0\rangle$};
  \node (0) at (469.0bp,162.0bp) [reach,nd0] {$s_0 = \langle a_0,b_0,c_0\rangle$};
  \node (4) at (469.0bp,234.0bp) [reach,nd5] {$\langle a_0,b_1,c_0\rangle$};
 
 
  \draw [->,arc3] (0) ..controls (445.65bp,135.46bp) and (435.95bp,124.86bp)  .. (2);
  \draw [->,arc5] (0) ..controls (475.71bp,188.03bp) and (475.94bp,197.36bp)  .. (4);
  \draw [->,arc4] (2) ..controls (405.0bp,63.983bp) and (405.0bp,54.712bp)  .. (6);
  \draw [->,arc1] (0) ..controls (499.93bp,134.79bp) and (517.41bp,122.58bp)  .. (1);
  \draw [->,arc2] (1) ..controls (539.0bp,117.26bp) and (521.52bp,129.47bp)  .. (0);
  \draw [->,arc5] (4) ..controls (462.3bp,208.35bp) and (462.06bp,199.03bp)  .. (0);
 
%
\end{tikzpicture}


}
}

\end{figure}

\end{column}
\end{columns}
%\medskip

\begin{liste}
  \item \tval{Automate}: composants \qex{$a$, $b$, $c$}.
\pause[2]
  \item \tval{\'Etats locaux}: niveaux d'expression \qex{$c_0$, $c_1$, $c_2$}.
\pause[3]
  \item \tval{\'Etats}: l'ensemble des états locaux actifs%
  \only<3-4>{\qex{$\PHetat{a_0, b_0, c_0}$}}%
  \only<5>{\qex{$\PHetat{a_1, b_0, c_0}$}}%
  \only<6>{\qex{$\PHetat{a_0, b_0, c_0}$}}%
  \only<7>{\qex{$\PHetat{a_2, b_0, c_0}$}}%
  \only<8>{\qex{$\PHetat{a_2, b_1, c_0}$}}
\pause[4]
  \item \tval{Transitions}: dynamique  \qex{\only<5>{\underline}{$t_1 = \trans{a_0}{a_1}{b_0}$}, \only<6>{\underline}{$t_2 = \trans{a_1}{a_0}{}$}, \only<7>{\underline}{$t_3 = \trans{a_0}{a_2}{b_0,c_0}$}, \only<8>{\underline}{$t_4 = \trans{b_0}{b_1}{}$}}
\end{liste}
%une seule transition à la fois
%on peut avoir un choix il faut bien expliquer ça.
\end{frame}

