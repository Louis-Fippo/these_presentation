\begin{frame}[c]
  \frametitle{Synthèse sur l'identification des bifurcations}

%mots clés: large scale
%bien expliquer les résultats obtenus
% \textbf{Summary}

%\pause
\tval{Approximations formelles des états et transitions de bifurcation}\\
\medskip
\textbf{Combinaison de deux abstractions}
  \begin{itemize}
   \item $$
\left\{
    \begin{array}{lcl}
        \mbox{PL (Programmation logique)} &\\  %rajouter un signe entre les deux
        \mbox{+} & \Rightarrow \mbox{\tval{approximations formelles des bifurcations}} \\
        \mbox{IA (Interprétation abstraite)} &
    \end{array}
\right.
$$
\item Applicable sur les grands modèles.
\item Manquer certaines bifurcations.
  \end{itemize}
\medskip
\textbf{Application sur des modèles biologiques réels}
\begin{itemize}
 \item Lambda phage, EGF/TNF, Différenciation des lymphocytes T.
 \item Base pour la reprogrammation cellulaire.
\end{itemize}
\medskip
\textbf{Publication}
\begin{itemize}
 \footnotesize
 \item
Louis Fippo Fitime, Olivier Roux, Carito Guziolowski, Loïc Paulevé. \tscite{Identification of Bifurcations in Biological
Regulatory Networks using Answer-Set
Programming}. In \textit{12th International Workshop on Constraint-Based Methods for Bioinformatics (WCB'12)}. September 2016.

\end{itemize}

\end{frame}