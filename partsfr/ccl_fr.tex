% Performances et conclusion
\begin{frame}[c]
  \frametitle{Conclusions}

%mots clés: large scale
%bien expliquer les résultats obtenus
% \textbf{Summary}
{\small
\pause
\textbf{Modélisation hybride des réseaux RSTC}
  \begin{itemize}
   \item Formalisation des connaissances biologiques: Translation des motifs en AN
   \item Inférence et intégration des paramètres temporels et stochastiques
   \item Raffinement qualitatif et quantitatif de la dynamique
  \end{itemize}
\pause
\textbf{Analyse statique des propriétés quantitatives (probabilité/délai)}
\begin{itemize}
 \item 
  $$
\left\{
    \begin{array}{lcl}
        \mbox{QS (Classes d'\'Equivalences d'\'Etats)} &\\  %rajouter un signe entre les deux
        \mbox{+} & \Longrightarrow \mbox{\tval{approximations formelles}} \\
        \mbox{IA (Interprétation Abstraite)} & \mbox{\tval{des propriétés quantitatives}}
    \end{array}
\right.
$$ 
 \item Estimation efficace d'une borne inférieure de la probabilité et du délai de l'accessibilité 
 \item Faible complexité théorique
\end{itemize}


\pause
\textbf{Identification des Bifurcations}
 \begin{itemize}
\item 
$$
\left\{
    \begin{array}{lcl}
        \mbox{PL (Programmation logique)} &\\  %rajouter un signe entre les deux
        \mbox{+} & \Longrightarrow \mbox{\tval{approximation formelle des bifurcations}} \\
        \mbox{IA (Interprétation abstraite)} &
    \end{array}
\right.
$$ 
   \item Applicable aux grands modèles  (vs model-checking)
  % \item Sous-approximation: possible de manquer certaines bifurcations.
   %\item Application on \tval{Phage lambda, EGF/TNF, Th\_differentiation} 
  \end{itemize}
\medskip
}
\end{frame}

%rajouter les slides du titre



